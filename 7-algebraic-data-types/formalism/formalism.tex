\documentclass{article}

\usepackage{xparse}
\usepackage{fullpage}
\usepackage{amsmath, stmaryrd, amssymb, mathtools, mathpartir}

% fixes spacing between math operators and large paired delimiters
\let\originalleft\left
\let\originalright\right
\renewcommand{\left}{\mathopen{}\mathclose\bgroup\originalleft}
\renewcommand{\right}{\aftergroup\egroup\originalright}

%%%%%%%%%%%%%%%%%%%%%%%%%%%%%%%%%%%%%%%%%%%%%%%%%%%%%%%%%%%%%%%%%%%%%%%%%%%%%%%%
% Syntax

% Sorts

\DeclareDocumentCommand{\HTyp}{}{\mathsf{HTyp}}
\DeclareDocumentCommand{\HExp}{}{\mathsf{HExp}}
\DeclareDocumentCommand{\HTag}{}{\mathsf{Tag}}

% Types

\DeclareDocumentCommand{\rectype}{!O{\alpha} !O{\tau}}{\mu #1.#2}
\DeclareDocumentCommand{\sumtype}{m m m}{+\left\{#1\left(#2\right)\right\}_{#3}}
\DeclareDocumentCommand{\typevar}{}{\alpha}

% Tags

\DeclareDocumentCommand{\C}{}{\mathcal{C}}
\DeclareDocumentCommand{\Tag}{}{\mathbf{C}}
\DeclareDocumentCommand{\TagHole}{!O{u}}{\mathopen{}?^{#1}\mathclose{}}

% Expressions

\DeclareDocumentCommand{\roll}{O{\tau} m}{\normalfont\texttt{roll}^{#1}\left(#2\right)}
\DeclareDocumentCommand{\unroll}{m m}{\normalfont\texttt{unroll}^{#1}\left(#2\right)}
\DeclareDocumentCommand{\inj}{O{\ell} O{\tau} m}{\normalfont\texttt{inj}_{#1}\left(#3\right)}
\DeclareDocumentCommand{\casexp}{m m m m m}{\normalfont\texttt{case} \left(#1\right) \left\{#2\left(#3\right)\mapsto #4\right\}_{#5}}
\DeclareDocumentCommand{\hole}{!O{}}{\llparenthesis#1\rrparenthesis}

%%%%%%%%%%%%%%%%%%%%%%%%%%%%%%%%%%%%%%%%%%%%%%%%%%%%%%%%%%%%%%%%%%%%%%%%%%%%%%%%
% Semantics

% Rules

% static
\DeclareDocumentCommand{\TRoll}{}{\textsc{TRoll}}
\DeclareDocumentCommand{\TUnroll}{}{\textsc{TUnroll}}
\DeclareDocumentCommand{\TInj}{}{\textsc{TInj}}
\DeclareDocumentCommand{\TCase}{}{\textsc{TCase}}
\DeclareDocumentCommand{\TSInj}{}{\textsc{TSInj}}
\DeclareDocumentCommand{\TAInj}{}{\textsc{TAInj}}

% dynamic
\DeclareDocumentCommand{\VRoll}{}{\textsc{VRoll}}
\DeclareDocumentCommand{\VInj}{}{\textsc{VInj}}
\DeclareDocumentCommand{\SSRollC}{}{\textsc{SSRollC}}
\DeclareDocumentCommand{\SSUnrollC}{}{\textsc{SSUnrollC}}
\DeclareDocumentCommand{\SSUnrollR}{}{\textsc{SSUnrollR}}
\DeclareDocumentCommand{\SSInjC}{}{\textsc{SSInjC}}
\DeclareDocumentCommand{\SSCaseC}{}{\textsc{SSCaseC}}
\DeclareDocumentCommand{\SSCaseR}{}{\textsc{SSCaseR}}

% Judgments

\DeclareDocumentCommand{\hastype}{O{} m m}{#1 \vdash #2 : #3}
\DeclareDocumentCommand{\syntype}{O{} m m}{#1 \vdash #2 \Rightarrow #3}
\DeclareDocumentCommand{\anatype}{O{} m m}{#1 \vdash #2 \Leftarrow #3}
\DeclareDocumentCommand{\MatchedSumType}{m m}{#1 \blacktriangleright_{+} #2}
\DeclareDocumentCommand{\smallstep}{}{\rightarrow}
\DeclareDocumentCommand{\isConsistent}{m m}{#1 \sim #2}
\DeclareDocumentCommand{\Join}{m m}{\join\left(#1,#2\right)}

% Operations

\DeclareDocumentCommand{\subst}{m m m}{[#1 / #2]#3}
\DeclareDocumentCommand{\defeq}{}{\stackrel{\text{def}}{=}}
\DeclareDocumentCommand{\isval}{m}{#1~\mathsf{\normalfont val}}

% Helpers

\DeclareMathOperator{\expand}{\text{\normalfont expand}}
\DeclareMathOperator{\join}{\text{\normalfont join}}

%%%%%%%%%%%%%%%%%%%%%%%%%%%%%%%%%%%%%%%%%%%%%%%%%%%%%%%%%%%%%%%%%%%%%%%%%%%%%%%%
% Theorems and Proofs

\usepackage{amsthm}
\newtheorem{Theorem}{Theorem}
\newtheorem{Lemma}[Theorem]{Lemma}

\usepackage{pfsteps}
\renewcommand\byCasesEveryCase{}
\setlength{\proofrightwidth}{0.4\linewidth}

%%%%%%%%%%%%%%%%%%%%%%%%%%%%%%%%%%%%%%%%%%%%%%%%%%%%%%%%%%%%%%%%%%%%%%%%%%%%%%%%

\title{Algebraic Data Types for Hazel}
\author{Eric Griffis \\ egriffis@umich.edu}
\date{}

\begin{document}

\maketitle

%%%%%%%%%%%%%%%%%%%%%%%%%%%%%%%%%%%%%%%%%%%%%%%%%%%%%%%%%%%%%%%%%%%%%%%%%%%%%%%%

\section{Formalism}

\subsection{Syntax}

\[
  \arraycolsep=0pt
  \begin{array}{l@{~~}c@{~~}c@{~~}ll}
    \HTyp & \tau & {}\Coloneqq{} & \ldots \mid (\rectype) \mid \typevar \mid \sumtype{C_i}{\tau_i}{C_i \in \C}                        \\
    \HExp & e    & {}\Coloneqq{} & \ldots \mid \roll{e} \mid \unroll{\tau}{e} \mid \inj{e} \mid \casexp{e}{C_i}{x_i}{e_i}{C_i \in \C} \\
    \HTag & C    & {}\Coloneqq{} & \Tag \mid \TagHole                                                                                 \\
  \end{array}
\]

\subsection{Static Semantics}

$\boxed{\hastype[\Gamma]{e}{\tau}}$

\begin{mathpar}
  \inferrule[TRoll]{
    \hastype[\Gamma]{e}{\expand(\tau)}
  }{
    \hastype[\Gamma]{\roll{e}}{\tau}
  }

  \inferrule[TUnroll]{
    \hastype[\Gamma]{e}{\tau}
  }{
    \hastype[\Gamma]{\unroll{\tau}{e}}{\expand(\tau)}
  }

  \inferrule[TInj]{
    \hastype[\Gamma]{e}{\tau_i}
  }{
    \hastype[\Gamma]{\inj[C_i]{e}}{\sumtype{C_i}{\tau_i}{C_i \in \C}}
  }

  \inferrule[TCase]{
    \hastype[\Gamma]{e}{\sumtype{C_i}{x_i}{C_i \in \C}} \\
    \hastype[\Gamma, x_i : \tau_i]{e_i}{\tau} \\
    i = 1, \ldots, N
  }{
    \hastype[\Gamma]{\casexp{e}{C_i}{x_i}{e_i}{C_i \in \C}}{\tau}
  }

  \inferrule[TCaseU]{
    \hastype[\Gamma]{e}{\hole} \\
    \hastype[\Gamma, x_i : \hole]{e_i}{\tau} \\
    i = 1, \ldots, N
  }{
    \hastype[\Gamma]{\casexp{e}{C_i}{x_i}{e_i}{C_i \in \C}}{\tau}
  }
\end{mathpar}
where $\expand(\rectype) \defeq \subst{\rectype}{\alpha}{\tau}$

% % % % % % % % % % % % % % % % % % % % % % % % % % % % % % % % % % % % % % % % 

\noindent
$\boxed{\syntype[\Gamma]{e}{\tau}}$

\begin{mathpar}
  \inferrule[Injection in Synthetic Position or Analyzed Against Non-Sum]{
    \anatype[\Gamma]{e}{\hole}
  }{
    \syntype[\Gamma]{\hole[\inj[C]{e}]}{\hole}
  }
\end{mathpar}

% % % % % % % % % % % % % % % % % % % % % % % % % % % % % % % % % % % % % % % % 

\noindent
$\boxed{\anatype[\Gamma]{e}{\tau}}$

\begin{mathpar}
  \inferrule[\TAInj]{
    C_j \in \C
    \\
    \anatype[\Gamma]{e}{\tau_j}
  }{
    \anatype[\Gamma]{\inj[C_j]{e}}{\sumtype{C_i}{\tau_i}{C_i \in \C}}
  }

  \inferrule[Bad Tag]{
    C \notin \C
    \\
    \anatype[\Gamma]{e}{\hole}
  }{
    \anatype[\Gamma]{\hole[\inj[C]{e}]}{\sumtype{C_i}{\tau_i}{C_i \in \C}}
  }

  \inferrule[Expected Type is Hole]{
    \anatype[\Gamma]{e}{\hole}
  }{
    \anatype[\Gamma]{\inj[C]{e}}{\hole}
  }
\end{mathpar}

% % % % % % % % % % % % % % % % % % % % % % % % % % % % % % % % % % % % % % % % 

\noindent
$\boxed{\isConsistent{\tau}{\tau'}}$

\begin{mathpar}
  \inferrule{
    \isConsistent{\tau_i}{\tau_i'}
  }{
    \isConsistent{
      \sumtype{C_i}{\tau_i}{C_i \in \C}
    }{
      \sumtype{C_i}{\tau_i'}{C_i \in \C}
    }
  }
\end{mathpar}

% % % % % % % % % % % % % % % % % % % % % % % % % % % % % % % % % % % % % % % % 

\noindent
$\boxed{\Join{\tau_1}{\tau_2} = \tau}$

\begin{mathpar}
  \Join{
    \sumtype{C_i}{\tau_i }{C_i \in \C}
  }{
    \sumtype{C_i}{\tau_i'}{C_i \in \C}
  }
  = \sumtype{C_i}{\Join{\tau_i}{\tau_i'}}{C_i \in \C}
\end{mathpar}




















\newpage

%%%%%%%%%%%%%%%%%%%%%%%%%%%%%%%%%%%%%%%%%%%%%%%%%%%%%%%%%%%%%%%%%%%%%%%%%%%%%%%%

\subsection{Dynamic Semantics}

$\boxed{\isval{e}}$

\begin{mathpar}
  \inferrule[VRoll]{
    \isval{e}
  }{
    \isval{\roll{e}}
  }

  \inferrule[VInj]{
    \isval{e}
  }{
    \isval{\inj[\ell_i]{e}}
  }
\end{mathpar}

\newpage

\noindent
$\boxed{e \smallstep e'}$

\begin{mathpar}
  \inferrule[SSRollC]{
    e \smallstep e'
  }{
    \roll{e} \smallstep \roll{e'}
  }

  \inferrule[SSUnrollC]{
    e \smallstep e'
  }{
    \unroll{\tau}{e} \smallstep \unroll{\tau}{e'}
  }

  \inferrule[SSUnrollR]{
    \isval{e}
  }{
    \unroll{\tau}{\roll{e}} \smallstep e
  }

  \inferrule[SSInjC]{
    e \smallstep e'
  }{
    \inj{e} \smallstep \inj{e'}
  }

  \inferrule[SSCaseC]{
    e \smallstep e'
  }{
    % \casexp{e}{C_i}{x_i}{e_i} \smallstep \casexp{e'}{C_i}{x_i}{e_i}
  }

  \inferrule[SSCaseR]{
    \isval{e}
  }{
    % \casexp{\inj[\ell_k]{e}}{C_i}{x_i}{e_i} \smallstep \subst{e}{x_k}{e_k}
  }
\end{mathpar}

Canonical forms theorem: if we have a closed, well-typed expression and it's a value,
then if $\tau$ is a sum type, it has certain form, and if it's a recursive type, it has a certain form.

\resetpfcounter
\begin{Lemma}[Canonical Forms]
  \label{thm:canonical forms}
  If $\hastype{e}{\tau}$
  and $\isval{e}$
  then all of the following are true:
  \begin{enumerate}
    \item
          If $\tau = \left( \rectype[\alpha][\tau'] \right)$,
          then $e = \roll{e'}$
          and $\hastype{e'}{\expand(\tau)}$
          and $\isval{e'}$.
    \item
          If $\tau = \left( \sumtype{C_i}{x_i}{C_i \in \C} \right)$,
          then $e = \inj[\ell_k]{e'}$ for some $\ell_k \in \Lambda$
          and $\hastype{e'}{\tau_k}$
          and $\isval{e'}$.
  \end{enumerate}
\end{Lemma}

\begin{proof}
  By rule induction on the static semantics:
  \begin{byCases}
    \case{\TRoll}
    By assumption, $e = $
    %
    \case{\TUnroll}
  \end{byCases}
  % By rule induction on the static semantics:
  % \begin{byCases}
  %   \case{\TRoll}
  %     \begin{pfsteps}
  %       \item A
  %       \item B
  %     \end{pfsteps}
  %   \case{\TUnroll}
  %     \begin{pfsteps}
  %       \item A
  %       \item B
  %     \end{pfsteps}
  %   \end{byCases}
\end{proof}



% If $\itgoes{e : \tau}$ and $\isval{e}$, then all of the following are true:
% \begin{enumerate}
%   \item If $\tau = (\mu \alpha.\tau')$, then $e = \roll{e'}$ and $\vdash e' : \expand(\tau)$ and $\isval{e'}$.
%   \item If $\tau = (\sum_{\ell_i \in \Lambda} \ell_i \mapsto \tau_i)$, then $e = \inj[\ell_k]{e'}$ for some $\ell_k \in \Lambda$ and $\vdash e' : \tau_k$ and $\isval{e'}$.
% \end{enumerate}
% (use rule induction to prove these)

% \subsection{Progress \& Preservation}

\noindent
progress = a well-typed expression is either a value or on its way to becoming one

\resetpfcounter
\begin{Theorem}[Progress]
  % \label{thm:progress}
  If \usepfcounter[e types]~$\hastype{e}{\tau}$
  then either $e \smallstep e'$
  or $\isval{e'}$.
\end{Theorem}

\begin{proof}
  By rule induction on \pfref{e types}:
  \begin{byCases}
    \item[TUnroll]
  \end{byCases}
\end{proof}

\noindent
preservation = stepping preserves typing

\noindent
progress + preservation = well-typed expressions don't go wrong

% If $\vdash e : \tau$ then either $\isval{e}$ or not.
% % If so, then by canonical forms, $e$ is a value.
% % Otherwise:
% % \begin{itemize}
% %   \item If $e = \unroll^\tau(e)$, then by induction hypothesis, $e'$ is either a value or on its way to becoming a value. (need dynamic semantics to continue)
% %   \item $e = \casexp(e') \! \left\{ \sum_{\ell_i \in \Lambda} \ell_i(x_i) \mapsto e_i \right\}$
% % \end{itemize}

% \vspace*{\baselineskip}

\newpage

\resetpfcounter
\begin{Theorem}[Preservation]
  % \label{thm:preservation}
  If \usepfcounter[e types]~$\hastype{e}{\tau}$
  and \usepfcounter[e steps] $e \smallstep e'$
  then \usepfcounter[e' types] $\hastype{e'}{\tau}$.
\end{Theorem}

% TODO: change itgoes to hastype{Gamma}{e}{tau}
% TODO: fix "by assumptions"
% TODO: use syntactic equality instead of explicit substitution
% TODO: add inversion lemma (for intro forms) and use it in TUnroll > SSUnrollR (on 12)
% TODO: use more descriptive subscripts

inversion lemma: |- e : tau and e has an e' then e' : tau

% \begin{proof}
%   By rule induction on \pfref{e types}:
%   \begin{byCases}
%     \item[\TRoll]
%       \begin{pfsteps}
%         \item @{roll:e roll} e = \roll{e_1}
%         \item @{roll:e1 types} \hastype{e_1}{\expand(\tau)}
%       \end{pfsteps}
%       By rule induction on \pfref{e steps}.
%       The only possible case is \SSRollC:
%       \begin{pfsteps}
%         \item @{roll:e1 steps} e_1 \smallstep e_1'
%         \item @{roll:e roll steps} e' = \roll{e_1'}
%         \item @{roll:e1' types} \hastype{e_1'}{\expand(\tau)} \BY{IH on \pfref{roll:e1 types, roll:e1 steps}}
%         \item \hastype{\roll{e_1'}}{\tau} \BY{\TRoll{} on \pfref{roll:e1' types}}
%       \end{pfsteps}
%     \item[\TUnroll]
%       \begin{pfsteps}
%         \item @{unroll:e unroll} e = \unroll{\tau'}{e_1}
%         \item @{unroll:t unroll} \hastype{e}{\expand(\tau')}
%         \item @{unroll:e1 types} \hastype{e_1}{\tau'}
%       \end{pfsteps}
%       By rule induction on \pfref{e steps}.
%       The only possible cases are \SSUnrollC{} and \SSUnrollR:
%       \begin{byCases}
%         \item[\SSUnrollC]
%           \begin{pfsteps}
%             \item @{unrollC:e1 steps} e_1 \smallstep e_1' \BY{\SSUnrollC{} on \pfref{e steps, unroll:e unroll}}
%             \item @{unrollC:e unroll steps} \unroll{\tau'}{e_1} \smallstep \unroll{\tau'}{e_1'} \BY{\SSUnrollC{} on \pfref{unroll:e unroll, unrollC:e1 steps}}
%             \item @{unrollC:e1' types} \hastype{e_1'}{\tau'} \BY{IH on \pfref{unroll:e1 types} and \pfref{unrollC:e1 steps}}
%             \item \hastype{\unroll{\tau'}{e_1'} : \expand(\tau')} \BY{\TUnroll{} on \pfref{unrollC:e1' types}}
%           \end{pfsteps}
%         \item[\SSUnrollR]
%           \begin{pfsteps}
%             \item @{unrollR:e1 roll} e_1 = \roll[\tau']{e'} \BY{\SSUnrollR{} on \pfref{e steps, unroll:e unroll}}
%             % \item @{unrollR:e1 val} \isval{e'} \BY{\SSUnrollR{} on \pfref{e steps, unrollR:e1 roll}}
%             % \item @{unrollR:e unroll steps} \unroll[\tau']{\roll[\tau']{e_{11}}} \smallstep e_{11} \BY{\SSUnrollR{} on \pfref{unrollR:e1 val}}
%             \item @{unrollR:e11 types} \hastype{e'}{\expand(\tau')} \BY{\TRoll{} on \pfref{unrollR:e1 roll, unroll:e1 types}}
%           \end{pfsteps}
%       \end{byCases}
%     \item[\TInj]
%       \begin{pfsteps}
%         \item @{inj:e inj} e = \inj[\ell_i]{e_1} \BY{\TUnroll{} on \pfref{e types}}
%         \item @{inj:t sum} \tau = \sumtype{C_i}{x_i}{C_i \in \C} \BY{\TUnroll{} on \pfref{e types}}
%         \item @{inj:e1 types} \hastype{e_1}{\tau_i} \BY{\TUnroll{} on \pfref{e types}}
%       \end{pfsteps}
%       By rule induction on \pfref{e steps}.
%       The only possible case is \SSInjC:
%       \begin{pfsteps}
%         \item @{inj:e steps} \inj[\ell_i]{e_1} \smallstep \inj[\ell_i]{e_1'} \BY{\SSInjC{} on \pfref{e steps, inj:e inj}}
%         \item @{inj:e1 steps} e_1 \smallstep e_1' \BY{\SSInjC{} on \pfref{inj:e inj, inj:e steps}}
%         \item @{inj:e1' types} \hastype{e_1'}{\tau_i} \BY{IH on \pfref{inj:e1 types, inj:e1 steps}}
%         \item @{inj:e' types} \hastype{\inj[\ell_i]{e_1'}}{\tau} \BY{\TInj{} on \pfref{inj:t sum, inj:e1' types}}
%       \end{pfsteps}
%     \item[\TCase]
%       \begin{pfsteps}
%         \item @{case:e case} e = \casexp{e_1}{C_i}{x_i}{e_{1i}} \BY{\TCase{} on \pfref{e types}}
%         \item @{case:e1 types} \hastype{e_1 : \sumtype{C_i}{x_i}{C_i \in \C}} \BY{\TCase{} on \pfref{e types}}
%         \item @{case:e1i types} \left\{ \hastype[x_i : \tau_i]{e_{1i} : \tau} \right\}_{i = 1}^N \BY{\TCase{} on \pfref{e types}}
%       \end{pfsteps}
%       By rule induction on \pfref{e steps}.
%       The only possible cases are \SSCaseC{} and \SSCaseR:
%       \begin{byCases}
%         \item[\SSCaseC]
%           \begin{pfsteps}
%             \item @{caseC:e1 steps} e_\text{scrut} \smallstep e_\text{scrut}'
%             \item @{caseC:e steps} e' = \casexp{e_\text{scrut}'}{C_i}{x_i}{e_{1i}}
%             \item @{caseC:e1' types} \hastype{e_\text{scrut}'}{\sumtype{C_i}{x_i}{C_i \in \C}} \BY{IH on \pfref{caseC:e1 steps, case:e1 types}}
%             \item @{caseC:e' types} \hastype{\casexp{e_\text{scrut}'}{C_i}{x_i}{e_{10}i} : \tau} \BY{\TCase{} on \pfref{caseC:e1' types, case:e1i types}}
%           \end{pfsteps}
%         \item[\SSCaseR]
%           \begin{pfsteps}
%             \item @{caseR:e1 inj} e_1 = \inj[\ell_k]{e_2} \BY{\SSCaseR{} on \pfref{e steps, case:e case}}
%             % \item @{caseR:e2 val} \isval{e_2} \BY{\SSCaseR{} on \pfref{case:e case, caseR:e1 inj}}
%             \item @{caseR:e steps} \casexp{\inj[\ell_k]{e_2}}{C_i}{x_i}{e_{1i}} \smallstep \subst{e_2}{x_k}{e_{1k}} \BY{\SSCaseR{} on \pfref{caseR:e2 val}}
%             \item @{caseR:e' types} \hastype{\subst{e_2}{x_k}{e_{1k}}}{\tau} \BY{SL on \pfref{case:e1i types, caseR:e2 val, }}
%           \end{pfsteps}
%       \end{byCases}
%   \end{byCases}
% \end{proof}

TODO: define (closed) substitution

(Need $e'$ to be closed to avoid unintended variable capture)

\DeclareDocumentCommand{\condSubst}{m m m m}{\left[ #1 / #2 \neq #3 \right] #4}

% $\begin{array}{ll}
%   \subst{e'}{x}{\roll{e}} & = \roll{\subst{e'}{x}{e}} \\
%   \subst{e'}{x}{\casexp{e}{C_i}{x_i}{e_i}} & = \casexp{\subst{e'}{x}{e}}{C_i}{x_i}{\condSubst{e'}{x}{x_i}{e_i}}
% \end{array}$

% TODO:
% \condSubst{[e'/x \neq x']e = [e'/x]e    when x \neq x'
% [e'/x \neq x']e = e          when x = x'

\resetpfcounter
\begin{Lemma}[Substitution]
  \label{thm:preservation}
  If \usepfcounter[x types e] $\hastype[x : \tau']{e}{\tau}$
  % and \usepfcounter[ei isval] $\isval{e_i}$
  and \usepfcounter[e' types] $\hastype{e'}{\tau'}$
  then \usepfcounter[e+e' types] $\hastype{\subst{e'}{x}{e} : \tau}$.
\end{Lemma}

\begin{proof}
  By induction on \pfref{x types e}.
  % (Then, by induction on the small step semantics)
\end{proof}


% Preservation: (typing is preserved by stepping)

% If (a) $\vdash e : \tau$ and (b) $e \smallstep e'$, then $\vdash e' : \tau$.

% \begin{itemize}
%   \item \TRoll:
%     $e = \roll{e_1}$
%     then (c) $\itgoes[]{e_1 : \expand(\tau)}$
%     \begin{itemize}
%       \item \SSRollC:
%         then (d) $e_1 \smallstep e_1'$
%         and (e) $e \smallstep \roll{e_1'}$
%         and by inductive hypothesis on (c) and (d):
%           (f) $\itgoes[]{e_1' : \expand(\tau)}$
%         so $\itgoes[]{\roll{e_1'} : \tau}$
%         by \TRoll on (f).
%       \item No other rules apply by syntactic contradiction
%     \end{itemize}
%   \item \TUnroll ...
%   \item \TInj ...
%   \item \TCase involves a lemma that says something about substitution (look up the substitution theorem if I get stuck)
% \end{itemize}

%%%%%%%%%%%%%%%%%%%%%%%%%%%%%%%%%%%%%%%%%%%%%%%%%%%%%%%%%%%%%%%%%%%%%%%%%%%%%%%%

% \newpage

% \bibliographystyle{amsplain}
% \bibliography{citations}

\end{document}