\documentclass{article}

\usepackage{fullpage}
\usepackage{amsmath, stmaryrd, amssymb, mathtools, mathpartir}

% Sorts

\newcommand{\IHExp}{\mathsf{IHExp}}
\newcommand{\IHTagExp}{\mathsf{IHTagExp}}
\newcommand{\HExp}{\mathsf{HExp}}
\newcommand{\HTyp}{\mathsf{HTyp}}
\newcommand{\HTag}{\mathsf{HTag}}
\newcommand{\HTagExp}{\mathsf{HTagExp}}
\newcommand{\HTagTyp}{\mathsf{HTagTyp}}
\newcommand{\HTagVar}{\mathsf{HTagVar}}

% Types

\newcommand{\rectype}[2]{\mu #1.#2}
\newcommand{\sumtype}[1]{+\mathopen{}\left\{#1\right\}}

% Tags

\newcommand{\C}{\mathcal{C}}
\newcommand{\Tag}{\mathbf{C}}
\newcommand{\TagHole}[1][u]{{?^{#1}}}
\newcommand{\Nothing}{\varnothing}

% Expressions

\newcommand{\roll}[2]{\mathtt{roll}^{#1}\mathopen{}\left(#2\right)\mathclose{}}
\newcommand{\unroll}[2]{\mathtt{unroll}^{#1}\mathopen{}\left(#2\right)\mathclose{}}
\newcommand{\inj}[3][]{\mathtt{inj}_{#2}^{#1}\mathopen{}\left(#3\right)\mathclose{}}
\newcommand{\case}[2]{\mathtt{case} \left(#1\right) \left\{#2\right\}}
\newcommand{\hole}[1][]{\llparenthesis#1\rrparenthesis}

% Judgments

\newcommand{\syn}[3]{#1 \vdash #2 \Rightarrow #3}
\newcommand{\ana}[3]{#1 \vdash #2 \Leftarrow #3}
\newcommand{\synelab}[5]{#1 \vdash #2 \Rightarrow #3 \leadsto #4 \dashv #5}
\newcommand{\anaelab}[6]{#1 \vdash #2 \Leftarrow #3 \leadsto #4 : #5 \dashv #6}
\newcommand{\anaE}[3]{#1 \vdash_E #2 \Leftarrow #3}
\newcommand{\assign}[3]{#1 \vdash #2 : #3}
\newcommand{\valid}[2]{#1 \vdash #2~\text{valid}}

% Operations

\newcommand{\subst}[3]{[#1 / #2]#3}
\newcommand{\defeq}{\stackrel{\text{def}}{=}}
\newcommand{\isval}[1]{#1~\mathsf{\normalfont val}}
\newcommand{\cast}[3]{#1\langle#2\,{\Rightarrow}\,#3\rangle}

% Helpers

\DeclareMathOperator{\expand}{\text{\normalfont expand}}
\DeclareMathOperator{\join}{\text{\normalfont join}}

%%%%%%%%%%%%%%%%%%%%%%%%%%%%%%%%%%%%%%%%%%%%%%%%%%%%%%%%%%%%%%%%%%%%%%%%%%%%%%%%

\title{Algebraic Data Types for Hazel}
\author{Eric Griffis \\ egriffis@umich.edu}
\date{}

\begin{document}

\maketitle

%%%%%%%%%%%%%%%%%%%%%%%%%%%%%%%%%%%%%%%%%%%%%%%%%%%%%%%%%%%%%%%%%%%%%%%%%%%%%%%%

\section{Syntax}

\[
  \arraycolsep=0pt  \arraycolsep=0pt
  \begin{array}{l@{~~}c@{~~}c@{~~}l}
    \IHExp & d & {}\Coloneqq{} &
      \ldots
      \mid \roll{\tau}{d}
      \mid \unroll{\tau}{d}
      \mid \inj[\tau]{C}{D}
      \mid \case{d}{C_i(X_i){=}d_i}_{C_i \in \C}
    \\
    \HExp & e & {}\Coloneqq{} &
      \ldots
      \mid \roll{\tau}{e}
      \mid \unroll{\tau}{e}
      \mid \inj{C}{E}
      \mid \case{e}{C_i(X_i){=}e_i}_{C_i \in \C}
    \\
    \HTyp & \tau & {}\Coloneqq{} &
      \ldots
      \mid (\rectype{\alpha}{\tau})
      \mid \alpha
      \mid \sumtype{C_i(T_i)}_{C_i \in \C}
    \\
    \HTag & C & {}\Coloneqq{} &
      \Tag
      \mid \TagHole
    \\
    \IHTagExp & D & {}\Coloneqq{} & d \mid \Nothing \\
    \HTagExp & E & {}\Coloneqq{} & e \mid \Nothing \\
    \HTagTyp & T & {}\Coloneqq{} & \tau \mid \Nothing \\
    \HTagVar & X & {}\Coloneqq{} & x \mid \Nothing \\
  \end{array}
\]

%%%%%%%%%%%%%%%%%%%%%%%%%%%%%%%%%%%%%%%%%%%%%%%%%%%%%%%%%%%%%%%%%%%%%%%%%%%%%%%%

\section{Semantics}

% % % % % % % % % % % % % % % % % % % % % % % % % % % % % % % % % % % % % % % % 
%  Gamma |- e => tau

\noindent
$\boxed{\syn{\Gamma}{e}{\tau}}$
\quad $e$ synthesizes type $\tau$
%
\begin{mathpar}
  \inferrule[SInj]{
    \valid{\Gamma}{E}
  }{
    \syn{\Gamma}{\hole[\inj{C}{E}]}{\hole}
  }
\end{mathpar}

\textsc{SInj} = Injection in Synthetic Position or Analyzed Against Non-Sum

\vspace*{\baselineskip}

% % % % % % % % % % % % % % % % % % % % % % % % % % % % % % % % % % % % % % % % 
%  Gamma |- E valid

\noindent
$\boxed{\valid{\Gamma}{E}}$
\quad $E$ is a valid expression
%
\begin{mathpar}
  \inferrule[]{
    \ana{\Gamma}{e}{\hole}
  }{
    \valid{\Gamma}{e}
  }

  \inferrule[]{
  }{
    \valid{\Gamma}{\Nothing}
  }
\end{mathpar}

\vspace*{\baselineskip}

% % % % % % % % % % % % % % % % % % % % % % % % % % % % % % % % % % % % % % % % 
% Gamma |- E <= T

\noindent
$\boxed{\anaE{\Gamma}{E}{T}}$
\quad $E$ analyzes against optional type $T$
%
\begin{mathpar}
  \inferrule[]{
    \ana{\Gamma}{e}{\tau}
  }{
    \anaE{\Gamma}{e}{\tau}
  }

  \inferrule[]{
  }{
    \anaE{\Gamma}{\Nothing}{\Nothing}
  }
\end{mathpar}

\vspace*{\baselineskip}

% % % % % % % % % % % % % % % % % % % % % % % % % % % % % % % % % % % % % % % % 
%  Gamma |- e <= tau

\noindent
$\boxed{\ana{\Gamma}{e}{\tau}}$
\quad $e$ analyzes against type $\tau$
%
\begin{mathpar}
  \inferrule[AInjHole]{
    \valid{\Gamma}{E}
  }{
    \ana{\Gamma}{\inj{C}{E}}{\hole}
  }

  \inferrule[AInj]{
    C_j \in \C \\
    \anaE{\Gamma}{E}{T_j}
  }{
    \ana{\Gamma}{\inj{C_j}{E}}{\sumtype{C_i(T_i)}_{C_i \in \C}}
  }

  \inferrule[AInjUnexpectedBody]{
    C_j \in \C \\
    T_j = \Nothing
  }{
    \ana{\Gamma}{\hole[\inj{C_j}{e}]}{\sumtype{C_i(T_i)}_{C_i \in \C}}
  }

  \inferrule[AInjExpectedBody]{
    C_j \in \C \\
    T_j = \tau
  }{
    \ana{\Gamma}{\hole[\inj{C_j}{\Nothing}]}{\sumtype{C_i(T_i)}_{C_i \in \C}}
  }

  \inferrule[AInjBadTag]{
    C \notin \C \\
    \valid{\Gamma}{E}
  }{
    \ana{\Gamma}{\hole[\inj{C}{E}]}{\sumtype{C_i(T_i)}_{C_i \in \C}}
  }
\end{mathpar}

\vspace*{\baselineskip}

% % % % % % % % % % % % % % % % % % % % % % % % % % % % % % % % % % % % % % % % 
% Gamma |- e => tau ~> d -| Delta

\noindent
$\boxed{\synelab{\Gamma}{e}{\tau}{d}{\Delta}}$
\quad $e$ synthesizes type $\tau$ and elaborates to $d$
%
\begin{mathpar}
  \inferrule{
    \anaelab{\Gamma}{e}{\hole}{d}{\tau}{\Delta}
  }{
    \synelab{\Gamma}{
      \hole[\inj{C}{e}]
    }{\hole}{
      \hole[\inj{C}{\cast{d}{\tau}{\hole}}]
    }{\Delta}
  }

  \inferrule{
  }{
    \synelab{\Gamma}{
      \hole[\inj{C}{\Nothing}]
    }{\hole}{
      \hole[{\inj[\Nothing]{C}{\Nothing}}]
    }{\emptyset}
  }
\end{mathpar}

\vspace*{\baselineskip}

% % % % % % % % % % % % % % % % % % % % % % % % % % % % % % % % % % % % % % % % 
% Gamma |- e <= tau_1 ~> d : tau_2 -| Delta

\noindent
$\boxed{\anaelab{\Gamma}{e}{\tau_1}{d}{\tau_2}{\Delta}}$
\quad $e$ analyzes against type $\tau_1$ and elaborates to $d$ of consistent type $\tau_2$
%
\begin{mathpar}
  \inferrule{
  }{
    \anaelab{\Gamma}{
      e
    }{\tau_1}{
      d
    }{\tau_2}{\Delta}
  }
\end{mathpar}

\vspace*{\baselineskip}

% % % % % % % % % % % % % % % % % % % % % % % % % % % % % % % % % % % % % % % % 
% Gamma |- e : tau

\noindent
$\boxed{\assign{\Gamma}{e}{\tau}}$
\quad $e$ is assigned type $\tau$
%
\begin{mathpar}
  \inferrule[TRoll]{
    \assign{\Gamma}{e}{\expand(\tau)}
  }{
    \assign{\Gamma}{\roll{\tau}{e}}{\tau}
  }

  \inferrule[TUnroll]{
    \assign{\Gamma}{e}{\tau}
  }{
    \assign{\Gamma}{\unroll{\tau}{e}}{\expand(\tau)}
  }

  \inferrule[TInj]{
    \assign{\Gamma}{E}{T_i}
  }{
    \assign{\Gamma}{\inj{C_i}{E}}{\sumtype{C_i(T_i)}_{C_i \in \C}}
  }

  \inferrule[TCase]{
    \assign{\Gamma}{e}{\sumtype{C_i(T_i)}_{C_i \in \C}} \\
    \assign{\Gamma, X_i : T_i}{e_i}{\tau} \\
    i = 1, \ldots, N
  }{
    \assign{\Gamma}{
      \case{e}{C_i(X_i){=}e_i}_{C_i \in \C}
    }{\tau}
  }

  \inferrule[TCaseU]{
    \assign{\Gamma}{e}{\hole} \\
    \assign{\Gamma, X_i : \hole}{e_i}{\tau} \\
    i = 1, \ldots, N
  }{
    \assign{\Gamma}{\case{e}{C_i(X_i){=}{e_i}}_{C_i \in \C}}{\tau}
  }
\end{mathpar}

$\expand(\rectype{\alpha}{\tau}) \defeq \subst{\rectype{\alpha}{\tau}}{\alpha}{\tau}$

\vspace*{\baselineskip}



% TODO: switch type assingment from e to d
% TODO: define elaboration judgments and write down the rules for them
% TODO: connect DHExp to paper theorems (sec 3)

%%%%%%%%%%%%%%%%%%%%%%%%%%%%%%%%%%%%%%%%%%%%%%%%%%%%%%%%%%%%%%%%%%%%%%%%%%%%%%%%

\end{document}