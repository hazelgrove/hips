\documentclass[12pt,fleqn]{article}
\usepackage[margin=2cm]{geometry}
\usepackage{titlesec} % see below
\usepackage{xcolor} % \color \textcolor
\usepackage{hyperref} % \href
\usepackage[normalem]{ulem} % normalem retains \emph as italic
% \uline \uuline \uwave \sout
\usepackage{enumitem}
% \begin{enumerate}[label=\Alpha*.]
% \begin{enumerate}[label=\(\bullet\)]
\usepackage{graphicx}
% \includegraphicswidth=0.5\textwidth,trim=[0cm 0cm 0cm 0cm,clip]{file.png}
\usepackage{pdfpages} % \includepdf[pages=1]{file.pdf}
% for prose
%\usepackage[doublespacing]{setspace}
\usepackage{csquotes} % \blockquote
%%%%%%%%%%%%%%%%%%%%%%%%%%%%%%%%%%%%%%%%%%%%%%%%%%%%%%%%%%%%
\renewcommand{\rmdefault}{cmr}
\renewcommand{\sfdefault}{cmss}
\renewcommand{\ttdefault}{cmtt}
\renewcommand{\familydefault}{\rmdefault}
\setlength{\titlewidth}{\textwidth}

\titlespacing{\section}{0pt}{0pt}{0pt}
\titlespacing{\subsection}{0pt}{0pt}{0pt}
\titlespacing{\subsubsection}{0pt}{0pt}{0pt}
\titlespacing{\paragraph}{0pt}{0pt}{0pt}
% https://www.overleaf.com/learn/latex/How_to_write_a_LaTeX_class_file_and_design_your_own_CV_(Part_1)
% \titleformat{command}[shape]{format}{label}{sep}{before-code}[after-code]
\titleformat{\section}         % Customise the \section command 
    [hang]
    {\Large\ttfamily\raggedright} % Make the \section headers large (\Large),
                               % small capitals (\scshape) and left aligned (\raggedright)
    {}{0em}                      % Can be used to give a prefix to all sections, like 'Section ...'
    {}                           % Can be used to insert code before the heading
    [\titlerule]                 % Inserts a horizontal line after the heading
\titleformat{\subsection}
    [hang]
    {\large\ttfamily\raggedright}
    {}{0em}
    {}
    []
%%%%%%%%%%%%%%%%%%%%%%%%%%%%%%%%%%%%%%%%%%%%%%%%%%%%%%%%%%%%
\usepackage{mathpartir}
\usepackage{latexsym}
\usepackage{stmaryrd}
\usepackage{amssymb}
\usepackage{thmtools}

%% Joshua Dunfield macros
\def\OPTIONConf{1}%
\usepackage{joshuadunfield}
%%%%%%%%%%%%%%%%%%%%%%%%%%%%%%%%%%%%%%%%%%%%%%%%%%%%%%%%%%%%
\definecolor{red}{HTML}{800000}
\definecolor{green}{HTML}{008000}
\definecolor{blue}{HTML}{000080}
\definecolor{purple}{HTML}{800080}
\definecolor{teal}{HTML}{008080}
\newcommand{\red}[1]{\textcolor{red}{#1}}
\newcommand{\green}[1]{\textcolor{green}{#1}}
\newcommand{\blue}[1]{\textcolor{blue}{#1}}
\newcommand{\purple}[1]{\textcolor{purple}{#1}}
\newcommand{\teal}[1]{\textcolor{teal}{#1}}
\newcommand{\redtt}[1]{\red{\mathtt{#1}}}
\newcommand{\greentt}[1]{\green{\mathtt{#1}}}
\newcommand{\bluett}[1]{\blue{\mathtt{#1}}}
\newcommand{\purplett}[1]{\purple{\mathtt{#1}}}
\newcommand{\tealtt}[1]{\teal{\mathtt{#1}}}
\newcommand{\redit}[1]{\red{\mathit{#1}}}
\newcommand{\greenit}[1]{\green{\mathit{#1}}}
\newcommand{\blueit}[1]{\blue{\mathit{#1}}}
\newcommand{\purpleit}[1]{\purple{\mathit{#1}}}
\newcommand{\tealit}[1]{\teal{\mathit{#1}}}

\newcommand{\knd}[1][]{\redit{knd#1}}
\newcommand{\typ}[1][]{\greenit{\tau#1}}
\newcommand{\typvar}[1][]{\greenit{\alpha#1}}
\newcommand{\modle}[1][]{\tealit{mod#1}}
\newcommand{\modlevar}[1][]{\tealit{\mu#1}}
\newcommand{\sig}[1][]{\purpleit{sig#1}}
\newcommand{\lab}[1][]{\mathit{lab#1}}
%%%%%%%%%%%%%%%%%%%%%%%%%%%%%%%%%%%%%%%%%%%%%%%%%%%%%%%%%%%%
\pagenumbering{gobble}
\nonfrenchspacing
%\frenchspacing % when monospaced
\begin{document}
\title{Hazel PHI: 10-modules}
\author{}
\date{}
\maketitle
\section{how to read}
\subsection*{}
\begin{tabular}{rlrl}
    \red{800000} & \red{kinds} & \purple{800080} & \purple{signatures} \\
    \green{008000} & \green{types (constructors)} & \teal{008080} & \teal{modules} \\
    \blue{000080} & \blue{terms} && \\
\end{tabular}
\section{syntax}
\[\begin{array}{rcrl}
    \mathrm{kind} & \knd & ::=
                  & \redtt{Type} \\
                  && \vert & \redtt{S(\typ)} \\
                  && \vert & \redtt{KHole} \\
                  && \vert & \redtt{\Pi_{\typvar::\knd[_1]}.\knd[_2]} \\
                  && \vert & \redtt{\Sigma_{\typvar::\knd[_1]}.\knd[_2]} \\
    \mathrm{base~type} & \greenit{bse} & ::=
                       & \greentt{Int} \\
                       && \vert & \greentt{Float} \\
                       && \vert & \greentt{Bool} \\
    \mathrm{HTyp~BinOp} & \greenit{\oplus} & ::=
                   & \greentt{\times} \\
                   && \vert & \greentt{+} \\
                   && \vert & \greentt{\rightarrow}\\
    \mathrm{internal~HTyp} & \typ & ::=
                           & \greenit{bse} \\
                           && \vert & \greentt{\lambda\typvar::\knd.\typ} \\
                           && \vert & \greentt{\typ[_1]~\typ[_2]} \\
                           && \vert & \greentt{\typ[_1] \oplus \typ[_2]} \\
                           && \vert & \greentt{\langle\typ[_1], \typ[_2]\rangle} \\
                           && \vert & \greentt{\pi_1~\typ} \\
                           && \vert & \greentt{\pi_2~\typ} \\
                           && \vert & \greentt{\{\lab[_1] \hookrightarrow \typ[_1], ...~\lab[_n] \hookrightarrow \typ[_n]\}} \\
                           && \vert & \greentt{\modle.\lab} \\
    \mathrm{module} & \modle & ::=
                    & [sbnd] \\
                    && \vert & \tealtt{\lambda\modlevar::\sig.\modle} \\
                    && \vert & \tealtt{\modle[_1]~\modle[_2]} \\
    \mathrm{signature} & \sig & ::=
                       & \\
                       && \vert & \purplett{\Pi_{\modlevar::\sig[_1]}.\sig[_2]} \\
\end{array}\]
\end{document}
