%%%%%%%%%%%%%%%%%%%%%%%%%%%%%%%%%%%%%%%%%%%%%%%%%%%%%%%%%%%%
% standard header
    \documentclass[10pt,fleqn]{article}
    \usepackage[top=2cm,bottom=2cm,left=1cm,right=1cm]{geometry}
    \usepackage{titlesec} % see below
    \usepackage{xcolor} % \color \textcolor
    \usepackage{hyperref} % \href
    \usepackage[normalem]{ulem} % normalem retains \emph as italic
    % \uline \uuline \uwave \sout
    \usepackage{enumitem}
    % \begin{enumerate}[label=\Alpha*.]
    % \begin{enumerate}[label=\(\bullet\)]
    \usepackage{graphicx}
    % \includegraphicswidth=0.5\textwidth,trim=[0cm 0cm 0cm 0cm,clip]{file.png}
    \usepackage{pdfpages} % \includepdf[pages=1]{file.pdf}
    % for prose
    %\usepackage[doublespacing]{setspace}
    \usepackage{csquotes} % \blockquote
    %%%%%%%%%%%%%%%%%%%%%%%%%%%%%%%%%%%%%%%%%%%%%%%%%%%%%%%%%%%%
    \renewcommand{\rmdefault}{cmr}
    \renewcommand{\sfdefault}{cmss}
    \renewcommand{\ttdefault}{cmtt}
    \renewcommand{\familydefault}{\rmdefault}
    \setlength{\titlewidth}{\textwidth}

    \titlespacing{\section}{0pt}{0pt}{0pt}
    \titlespacing{\subsection}{0pt}{0pt}{0pt}
    \titlespacing{\subsubsection}{0pt}{0pt}{0pt}
    \titlespacing{\paragraph}{0pt}{0pt}{0pt}
    % https://www.overleaf.com/learn/latex/How_to_write_a_LaTeX_class_file_and_design_your_own_CV_(Part_1)
    % \titleformat{command}[shape]{format}{label}{sep}{before-code}[after-code]
    \titleformat{\section}         % Customise the \section command 
        [hang]
        {\Large\ttfamily\raggedright} % Make the \section headers large (\Large),
                                   % small capitals (\scshape) and left aligned (\raggedright)
        {}{0em}                      % Can be used to give a prefix to all sections, like 'Section ...'
        {}                           % Can be used to insert code before the heading
        [\titlerule]                 % Inserts a horizontal line after the heading
    \titleformat{\subsection}
        [hang]
        {\large\ttfamily\raggedright}
        {}{0em}
        {}
        []
%%%%%%%%%%%%%%%%%%%%%%%%%%%%%%%%%%%%%%%%%%%%%%%%%%%%%%%%%%%%
% packages
    \usepackage{mathpartir}
    \usepackage{latexsym}
    \usepackage{stmaryrd}
    \usepackage{amssymb}
    \usepackage{thmtools}
    \usepackage{tikz}
    % https://tex.stackexchange.com/questions/12678/squiggly-arrows-in-tikz
    \usetikzlibrary{decorations.pathmorphing}
    \usetikzlibrary{arrows.meta}
    % https://tex.stackexchange.com/questions/317121/automatically-adding-page-breaks-into-long-array-environments
    \usepackage{array}
    \usepackage{longtable}
    \newcolumntype{L}{>{\(}l<{\)}}
    \newcolumntype{C}{>{\(}c<{\)}}
    \newcolumntype{R}{>{\(}r<{\)}}
    \setlength\tabcolsep{5pt}

    %% Joshua Dunfield macros
    \def\OPTIONConf{1}%
    \usepackage{joshuadunfield}
%%%%%%%%%%%%%%%%%%%%%%%%%%%%%%%%%%%%%%%%%%%%%%%%%%%%%%%%%%%%
% color defs
    \definecolor{red}{HTML}{800000}
    \definecolor{green}{HTML}{008000}
    \definecolor{blue}{HTML}{000080}
    \definecolor{orange}{HTML}{D08000}
    \definecolor{purple}{HTML}{800080}
    \definecolor{teal}{HTML}{008080}
    \definecolor{black}{HTML}{000000}
    \definecolor{grey}{HTML}{808080}
    \newcommand{\red}[1]{\textcolor{red}{#1}}
    \newcommand{\green}[1]{\textcolor{green}{#1}}
    \newcommand{\blue}[1]{\textcolor{blue}{#1}}
    \newcommand{\orange}[1]{\textcolor{orange}{#1}}
    \newcommand{\purple}[1]{\textcolor{purple}{#1}}
    \newcommand{\teal}[1]{\textcolor{teal}{#1}}
    \newcommand{\black}[1]{\textcolor{black}{#1}}
    \newcommand{\grey}[1]{\textcolor{grey}{#1}}
    \newcommand*{\blackrm}[1]{\black{\mathtt{#1}}}
    \newcommand{\redtt}[1]{\red{\mathtt{#1}}}
    \newcommand{\greentt}[1]{\green{\mathtt{#1}}}
    \newcommand{\bluett}[1]{\blue{\mathtt{#1}}}
    \newcommand{\orangett}[1]{\orange{\mathtt{#1}}}
    \newcommand{\purplett}[1]{\purple{\mathtt{#1}}}
    \newcommand{\tealtt}[1]{\teal{\mathtt{#1}}}
    \newcommand{\blacktt}[1]{\black{\mathtt{#1}}}
    \newcommand{\greytt}[1]{\grey{\mathtt{#1}}}
    \newcommand{\redit}[1]{\red{\mathit{#1}}}
    \newcommand{\greenit}[1]{\green{\mathit{#1}}}
    \newcommand{\blueit}[1]{\blue{\mathit{#1}}}
    \newcommand{\orangeit}[1]{\orange{\mathit{#1}}}
    \newcommand{\purpleit}[1]{\purple{\mathit{#1}}}
    \newcommand{\tealit}[1]{\teal{\mathit{#1}}}
    \newcommand{\blackit}[1]{\black{\mathit{#1}}}
    \newcommand{\greyit}[1]{\grey{\mathit{#1}}}
    % traditional hazel hole color
    \newcommand{\violettt}[1]{\textcolor{violet}{\mathtt{#1}}}
    \newcommand{\violetit}[1]{\textcolor{violet}{\mathit{#1}}}
%%%%%%%%%%%%%%%%%%%%%%%%%%%%%%%%%%%%%%%%%%%%%%%%%%%%%%%%%%%%
% syntax
    \newcommand{\knd}[1][]{\redit{\kappa#1}}
    \newcommand{\utyp}[1][]{\greenit{\hat{\tau#1}}}
    \newcommand{\typ}[1][]{\greenit{\tau#1}}
    \newcommand{\utypvar}[1][]{\greenit{t#1}}
    \newcommand{\typvar}[1][]{\greenit{t#1}}
    \newcommand{\uexp}[1][]{\blueit{\hat{\delta#1}}}
    \renewcommand{\exp}[1][]{\blueit{\delta#1}}
    \newcommand{\uexpvar}[1][]{\blueit{x#1}}
    \newcommand{\expvar}[1][]{\blueit{x#1}}
    \newcommand{\sigknd}[1][]{\orangeit{\xi#1}}
    \newcommand{\usig}[1][]{\purpleit{\hat{\sigma#1}}}
    \newcommand{\sig}[1][]{\purpleit{\sigma#1}}
    \newcommand{\usigvar}[1][]{\purpleit{s#1}}
    \newcommand{\sigvar}[1][]{\purpleit{s#1}}
    \newcommand{\umod}[1][]{\tealit{\hat{\mu#1}}}
    \renewcommand{\mod}[1][]{\tealit{\mu#1}}
    \newcommand{\umodvar}[1][]{\tealit{m#1}}
    \newcommand{\modvar}[1][]{\tealit{m#1}}
    \newcommand{\lab}[1][]{\mathit{lab#1}}

    \newcommand{\Type}{\redtt{Type}}
    \newcommand{\SKind}[1]{\redtt{S(#1)}}
    \newcommand{\KHole}{\redtt{KHole}}
    \newcommand{\DepFunKind}[2]{\redtt{\Pi_{#1}.{#2}}}
    \newcommand{\DepProdKind}[2]{\redtt{\Sigma_{#1}.{#2}}}

    \newcommand{\TypCFun}[2]{\greentt{\lambda #1.#2}}
    \newcommand{\TypCAp}[2]{\greentt{#1~#2}}
    \newcommand{\TypCPair}[2]{\greentt{\langle #1, #2\rangle}}
    \newcommand{\TypCPairPrjL}[1]{\greentt{\pi_1~#1}}
    \newcommand{\TypCPairPrjR}[1]{\greentt{\pi_2~#1}}
    \newcommand{\ListTyp}[1]{\greentt{[#1]}}
    % use violet instead of green?
    \newcommand{\TypCHole}[1][]{\greentt{\llparenthesis #1 \rrparenthesis}}
    \newcommand{\ModTypPrj}[2]{\greentt{#1.#2}}

    \newcommand{\Int}{\greentt{Int}}
    \newcommand{\Float}{\greentt{Float}}
    \newcommand{\Bool}{\greentt{Bool}}

    \newcommand{\uLetSig}[2]{\purplett{signature~#1=#2}\bluett{~in~\uexp}}
    \newcommand{\LetSig}[2]{\purplett{signature~#1=#2}\bluett{~in~\exp}}
    \newcommand{\uLetMod}[2]{\tealtt{module~#1=#2}\bluett{~in~\uexp}}
    \newcommand{\LetMod}[2]{\tealtt{module~#1=#2}\bluett{~in~\exp}}
    \newcommand{\uLetFunctor}[2]{\tealtt{functor~#1=#2}\bluett{~in~\uexp}}
    \newcommand{\LetFunctor}[2]{\tealtt{functor~#1=#2}\bluett{~in~\exp}}
    \newcommand{\uModTermPrj}[2]{\bluett{#1.#2}}
    \newcommand{\ModTermPrj}[2]{\bluett{#1.#2}}

    \newcommand*{\SSigKind}[1]{\orangett{SSigKind(#1)}}
    \newcommand*{\SigKHole}{\orangett{SigKHole}}

    \newcommand{\Sig}[1]{\purplett{\{#1\}}}
    \newcommand{\FunctorSig}[2]{\purplett{\Pi_{#1}.{#2}}}
    \newcommand{\SigHole}[1][]{\purplett{\llparenthesis#1\rrparenthesis}}

    \newcommand{\Struct}[1]{\tealtt{\{#1\}}}
    \newcommand{\Functor}[2]{\tealtt{\lambda #1.#2}}
    \newcommand{\FunctorAp}[2]{\tealtt{#1~#2}}
    \newcommand{\SubModulePrj}[2]{\tealtt{#1.#2}}
    \newcommand{\ModHole}[1][]{\tealtt{\llparenthesis#1\rrparenthesis}}


    \newcommand{\usbnd}{\greyit{\hat{sbnd}}}
    \newcommand{\sbnd}{\greyit{sbnd}}
    \newcommand{\usbnds}{\greyit{\hat{sbnds}}}
    \newcommand{\sbnds}{\greyit{sbnds}}
    \newcommand{\usdec}{\greyit{\hat{sdec}}}
    \newcommand{\sdec}[1][]{\greyit{sdec#1}}
    \newcommand{\usdecs}{\greyit{\hat{sdecs}}}
    \newcommand{\sdecs}[1][]{\greyit{sdecs#1}}

    \newcommand{\OpaqueTypeSdec}[1]{\greentt{type~#1}}
    \newcommand{\TransparentTypeSdec}[2]{\greentt{type~#1 = #2}}
    \newcommand{\ValSdec}[2]{\bluett{val~#1\TypAnn #2}}
    \newcommand{\ModSdec}[2]{\tealtt{module~#1\SigAnn #2}}
    \newcommand{\FunctorSdec}[2]{\tealtt{functor~#1\SigAnn #2}}

    \newcommand{\TypeSbnd}[2]{\greentt{type~#1 = #2}}
    \newcommand{\ValSbnd}[2]{\bluett{let~#1 = #2}}
    \newcommand{\ModSbnd}[2]{\tealtt{module~#1 = #2}}
    \newcommand{\FunctorSbnd}[2]{\tealtt{functor~#1 = #2}}
%%%%%%%%%%%%%%%%%%%%%%%%%%%%%%%%%%%%%%%%%%%%%%%%%%%%%%%%%%%%
% misc
    \newcommand*{\ExpVarCtx}{\Gamma}
    \newcommand*{\TypVarCtx}{\Phi}
    \newcommand*{\ModVarCtx}{\Xi}
    \newcommand*{\SigVarCtx}{\Psi}
    \newcommand*{\HoleCtx}{\Delta}
    \newcommand{\dom}[1]{\mathsf{dom(}#1\mathsf{)}}
    \newcommand{\val}[1]{\mathsf{val(}#1\mathsf{)}}
    \newcommand*{\type}[2]{\mathsf{type(}#1, #2\mathsf{)}}
    \newcommand{\submodule}[1]{\mathsf{submodule(}#1\mathsf{)}}
    \renewcommand*{\hole}[1][]{\blackrm{u#1}}

    \newcommand{\hintpagebreak}{\pagebreak[3]} % 0-4; 2 too weak, 4 too strong
%%%%%%%%%%%%%%%%%%%%%%%%%%%%%%%%%%%%%%%%%%%%%%%%%%%%%%%%%%%%
% judgements
    % declarative
        \newcommand{\TypAsc}{\black{:}}
        \newcommand{\TypAssump}{\black{:}}
        \newcommand{\TypAnn}{\black{:}}
        \newcommand{\KndAsc}{\black{::}}
        \newcommand{\KndAssump}{\black{::}}
        \newcommand{\KndAnn}{\black{::}} % why do we have this?
        \newcommand*{\SigKndAsc}{\black{::_{\sig}}}
        \newcommand*{\SigKndAssump}{\black{::_{\sig}}}
        \newcommand{\SigAsc}{\black{:_{\mod}}}
        \newcommand{\SigAssump}{\black{:_{\mod}}}
        \newcommand{\SigAnn}{\black{:_{\mod}}}

        \newcommand{\ConsistentSubkindOp}{\lesssim}
        \newcommand{\ConsistentSubkind}[3]{\mathrm{#1 \vdash #2 \ConsistentSubkindOp #3}}
        \newcommand{\ConsistentSubSigKindOp}{\lesssim_{\sigknd}}
        \newcommand{\ConsistentSubSigKind}[3]{\mathrm{#1 \vdash #2 \ConsistentSubSigKindOp #3}}
        \newcommand*{\SubSdecOp}{\le_{\sdec}}
        \newcommand*{\SubSdec}[3]{\mathrm{#1 \vdash #2 \SubSdecOp #3}}

        \newcommand*{\TypeEquiv}[3]{\mathrm{#1 \vdash #2 \equiv #3}}
        \newcommand*{\KindEquiv}[3]{\mathrm{#1 \vdash #2 \equiv #3}}
        \newcommand*{\SigEquiv}[3]{\mathrm{#1 \vdash #2 \equiv #3}}
        \newcommand*{\SigKindEquiv}[3]{\mathrm{#1 \vdash #2 \equiv #3}}
        \newcommand*{\WellFormedAtType}[3]{\mathrm{#1 \vdash #2 \TypAsc #3}}
        \newcommand*{\WellFormedAtKind}[3]{\mathrm{#1 \vdash #2 \KndAsc #3}}
        \newcommand*{\WellFormedAtSig}[3]{\mathrm{#1 \vdash #2 \SigAsc #3}}
        \newcommand*{\WellFormedAtSigKind}[3]{\mathrm{#1 \vdash #2 \SigKndAsc #3}}
    % algorithmic
        \newcommand{\Syn}[3]{#1 \vdash #2~\Rightarrow~#3}
        \newcommand{\Ana}[3]{#1 \vdash #2~\Leftarrow~#3}
        \newcommand{\SynElab}[5]{#1 \vdash #2~\Rightarrow~#3 \rightsquigarrow #4 \dashv #5}
        \newcommand{\AnaElab}[5]{#1 \vdash #2~\Leftarrow~#3 \rightsquigarrow #4 \dashv #5}
        \newcommand*{\ElabDegenerate}[4]{#1 \vdash #2 \rightsquigarrow #3 \dashv #4}
%%%%%%%%%%%%%%%%%%%%%%%%%%%%%%%%%%%%%%%%%%%%%%%%%%%%%%%%%%%%
\pagenumbering{gobble}
\nonfrenchspacing
%\frenchspacing % when monospaced
\begin{document}
\title{Hazel PHI: 10-modules}
\author{}
\date{\today}
\maketitle
\section{prerequisites}
    \subsection*{}
    \begin{itemize}
        \item Hazel PHI: 9-type-aliases-redux
            \begin{itemize}
                \item \href{https://github.com/hazelgrove/phi/tree/9-type-aliases-redux}{github}
                \item current commit: 4410cd565ce717707e580e44f64868d3175fe2a6
            \end{itemize}
        \item (optional) Hazel PHI: 1-labeled-tuples
            \begin{itemize}
                \item \href{https://github.com/hazelgrove/phi/tree/1-labeled-tuples}{github}
                \item current commit: 0a7d0b53ee7286d03ea3be13a7ac91a86f1c90b1
            \end{itemize}
    \end{itemize}
\section{how to read}
    \subsection*{}
    \begin{tabular}{rlrl}
        \red{800000} & \red{kinds} & \orange{D08000} & \orange{signature kind} \\
        \green{008000} & \green{types (constructors)} & \purple{800080} & \purple{signatures} \\
        \blue{000080} & \blue{terms} & \teal{008080} & \teal{modules} \\
    \end{tabular}
\section{notes}
    \subsection*{}
    \begin{tikzpicture}[
        kind/.style={circle, draw=red!100, fill=red!10, thick},
        type/.style={circle, draw=green!100, fill=green!10, thick},
        term/.style={circle, draw=blue!100, fill=blue!10, thick},
        signature kind/.style={rectangle, draw=orange!100, fill=orange!10, thick},
        signature/.style={rectangle, draw=purple!100, fill=purple!10, thick},
        module/.style={rectangle, draw=teal!100, fill=teal!10, thick},
        sdec/.style={rectangle, draw=grey!100, fill=grey!10, thick},
        sbnd/.style={rectangle, draw=grey!100, fill=grey!10, thick},
        ->/.style={line join=round, decorate, decoration={zigzag, segment length=4, amplitude=.9, post=lineto, post length=10pt}},
        note/.style={rectangle, thick},
        ]
        \node[kind](knd) at (0, 4){$\knd$};
        \node[type](utyp) at (-1, 2){$\utyp$};
        \node[type](typ) at (1, 2){$\typ$};
        \node[term](uexp) at (-1, 0){$\uexp$};
        \node[term](exp) at (1, 0){$\exp$};
        \node[signature kind](sigknd) at (4, 5){$\sigknd$};
        \node[signature](usig) at (3, 3){$\usig$};
        \node[signature](sig) at (5, 3){$\sig$};
        \node[module](umod) at (3, 1){$\umod$};
        \node[module](mod) at (5, 1){$\mod$};
        \node[sdec](usdec) at (7, 3){$\usdec$};
        \node[sdec](sdec) at (9, 3){$\sdec$};
        \node[sbnd](usbnd) at (7, 1){$\usbnd$};
        \node[sbnd](sbnd) at (9, 1){$\sbnd$};
        %\node[note] at (7, 2){$\utyp := \typ$};
        %\node[note] at (7, 3){$\usig := \sig$};
        %\node[note] at (7, 1){$\umod := \mod$};
        \node[note] at (0.7, 4){$\ConsistentSubkindOp$};
        \node[note] at (4.7, 5){$\ConsistentSubSigKindOp$};
        \node[note] at (10.1, 3){$\SubSdecOp$};
        \draw[->](utyp.east)--(typ.west);
        \draw[->](uexp.east)--(exp.west);
        \draw[->](usig.east)--(sig.west);
        \draw[->](umod.east)--(mod.west);
        \draw[->](usdec.east)--(sdec.west);
        \draw[->](usbnd.east)--(sbnd.west);
        \node[draw=none, shift={(-0.1, 0)}] at (typ.west){$>$};
        \node[draw=none, shift={(-0.1, 0)}] at (exp.west){$>$};
        \node[draw=none, shift={(-0.1, 0)}] at (sig.west){$>$};
        \node[draw=none, shift={(-0.1, 0)}] at (mod.west){$>$};
        \node[draw=none, shift={(-0.1, 0)}] at (sdec.west){$>$};
        \node[draw=none, shift={(-0.1, 0)}] at (sbnd.west){$>$};
        \draw[-{Implies}, double]([shift={(-0.1, 0)}]uexp.north)--([shift={(-0.1, 0)}]typ.south);
        \draw[-{Implies}, double]([shift={(0.1, 0)}]typ.south)--([shift={(0.1, 0)}]uexp.north);
        \draw[-{Implies}, double]([shift={(-0.1, 0)}]utyp.north)--([shift={(-0.1, 0)}]knd.south);
        \draw[-{Implies}, double]([shift={(0.1, 0)}]knd.south)--([shift={(0.1, 0)}]utyp.north);
        \draw[-{Implies}, double]([shift={(-0.1, 0)}]typ.north)--([shift={(-0.1, 0)}]knd.south);
        \draw[-{Implies}, double]([shift={(0.1, 0)}]knd.south)--([shift={(0.1, 0)}]typ.north);
        \draw[-{Implies}, double]([shift={(-0.1, 0)}]umod.north)--([shift={(-0.1, 0)}]sig.south);
        \draw[-{Implies}, double]([shift={(0.1, 0)}]sig.south)--([shift={(0.1, 0)}]umod.north);
        \draw[-{Implies}, double]([shift={(-0.1, 0)}]usig.north)--([shift={(-0.1, 0)}]sigknd.south);
        \draw[-{Implies}, double]([shift={(0.1, 0)}]sigknd.south)--([shift={(0.1, 0)}]usig.north);
        \draw[-{Implies}, double]([shift={(-0.1, 0)}]sig.north)--([shift={(-0.1, 0)}]sigknd.south);
        \draw[-{Implies}, double]([shift={(0.1, 0)}]sigknd.south)--([shift={(0.1, 0)}]sig.north);
        \draw[-{Implies}, double]([shift={(-0.1, 0)}]usbnd.north)--([shift={(-0.1, 0)}]sdec.south);
        \draw[-{Implies}, double]([shift={(0.1, 0)}]sdec.south)--([shift={(0.1, 0)}]usbnd.north);
        % https://tex.stackexchange.com/questions/71478/how-to-center-one-node-exactly-between-two-others-with-tikz
        \path (exp)--(typ) node[midway]{$\TypAsc$};
        \path (mod)--(sig) node[midway]{$\SigAsc$ ?};
    \end{tikzpicture}
    \subsection*{}
    external typ/sig/mod syntax not written out yet (waiting for construction dust to settle);
    patterns not handled yet-- will be left till end.
\section{syntax}
\begin{longtable}{RCRLR}
    \textrm{kind} & \knd & ::=
                  & \Type & \textrm{kind of types} \\
                  && \vert & \SKind{\typ} & \textrm{singleton kind} \\
                  && \vert & \KHole & \textrm{kind hole} \\
                  % functors?
                  && \vert & \DepFunKind{\typvar\KndAnn\knd[_1]}{\knd[_2]} & \textrm{dependent function kind} \\
                  %&& \vert & \DepProdKind{\typvar\KndAnn\knd[_1]}{\knd[_2]} & \textrm{dependent product kind} \\
    \hintpagebreak
    \textrm{HTyp} & \typ & ::=
                           & \typvar & \textrm{type variable} \\
                           && \vert & \greenit{bse} & \textrm{base type} \\
                           && \vert & \greentt{\typ[_1] \oplus \typ[_2]} & \textrm{type binop} \\
                           && \vert & \ListTyp{\typ} & \textrm{list type} \\
                           % functors?
                           && \vert & \TypCFun{\typvar\KndAnn\knd}{\typ} & \textrm{type function} \\
                           && \vert & \TypCAp{\typ[_1]}{\typ[_2]} & \textrm{type application} \\
                           %&& \vert & \TypCPair{\typ[_1]}{\typ[_2]} & \textrm{type pair} \\
                           %&& \vert & \TypCPairPrjL{\typ} & \textrm{type projection} \\
                           %&& \vert & \TypCPairPrjR{\typ} & \textrm{type projection} \\
                           % to compare against module term projection
                           && \vert & \greentt{\{\lab[_1] \hookrightarrow \typ[_1], ...~\lab[_n] \hookrightarrow \typ[_n]\}} & \textrm{labelled product type (record)} \\
                           && \vert & \ModTypPrj{\mod}{\lab} & \textrm{module type projection} \\
                           && \vert & \TypCHole & \textrm{empty type hole} \\
                           && \vert & \TypCHole[\typ] & \textrm{nonempty type hole} \\
    \hintpagebreak
    \textrm{base type} & \greenit{bse} & ::=
                       & \Int \\
                       && \vert & \Float \\
                       && \vert & \Bool \\
    \hintpagebreak
    \textrm{HTyp BinOp} & \greenit{\oplus} & ::=
                   & \greentt{\times} \\
                   && \vert & \greentt{+} \\
                   && \vert & \greentt{\rightarrow}\\
    \hintpagebreak
    \textrm{external expression} & \uexp & ::=
                                 & ... \\
                                 && \vert & \uexpvar \\
                                 && \vert & \uLetSig{\usigvar}{\usig} \\
                                 && \vert & \uLetMod{\umodvar}{\umod} \\
                                 && \vert & \uLetMod{\umodvar\SigAnn\usigvar}{\umod} \\
                                 && \vert & \uLetFunctor{something}{something} \\
                                 && \vert & \uModTermPrj{\umod}{\lab} & \textrm{module term projection} \\
    \hintpagebreak
    \textrm{internal expression} & \exp & ::=
                                 & ... \\
                                 && \vert &  \expvar \\
                                 && \vert & \LetSig{\sigvar}{\sig} \\
                                 && \vert & \LetMod{\modvar\SigAnn\sigvar}{\mod} \\
                                 && \vert & \LetFunctor{something}{something} \\
                                 && \vert & \ModTermPrj{\mod}{\lab} & \textrm{module term projection} \\
    \hintpagebreak
    \textrm{signature kind} & \sigknd & ::=
                            & \SSigKind{\sig} & \\
                            && \vert & \SigKHole & \\
    \hintpagebreak
    \textrm{signature} & \sig & ::=
                       & \sigvar & \textrm{signature variable} \\
                       && \vert & \Sig{\sdecs} & \textrm{structure signature} \\
                       && \vert & \FunctorSig{\modvar\SigAnn\sig[_1]}{\sig[_2]} & \textrm{functor signature} \\
                       && \vert & \SigHole & \textrm{empty signature hole} \\
                       && \vert & \SigHole[\sigvar] & \textrm{nonempty signature hole} \\
    \hintpagebreak
    \textrm{module} & \mod & ::=
                    & \modvar & \textrm{module variable} \\
                    && \vert & \Struct{\sbnds} & \textrm{structure} \\
                    && \vert & \Functor{\modvar\SigAnn\sig}{\mod} & \textrm{functor} \\
                    && \vert & \FunctorAp{\mod[_1]}{\mod[_2]} & \textrm{functor application} \\
                    && \vert & \SubModulePrj{\mod}{\lab} & \textrm{submodule projection} \\
                    && \vert & \ModHole & \textrm{empty module hole} \\
                    && \vert & \ModHole[\mod] & \textrm{nonempty module hole} \\
    \hintpagebreak
    \textrm{signature declarations} & \sdecs & ::=
                                & \cdot \\
                                && \vert & \sdec, \sdecs\\
    \hintpagebreak
    \textrm{signature declaration} & \sdec & ::=
                                   & \OpaqueTypeSdec{\lab} \\
                                   && \vert & \TransparentTypeSdec{\lab}{\typ} \\
                                   && \vert & \ValSdec{\lab}{\typ} \\
                                   && \vert & \ModSdec{\lab}{\sig} \\
                                   && \vert & \FunctorSdec{\lab}{\sig} \\
    \hintpagebreak
    \textrm{structure bindings} & \sbnds & ::=
                                & \cdot \\
                                && \vert & \sbnd, \sbnds \\
    \hintpagebreak
    \textrm{structure binding} & \sbnd & ::=
                               & \TypeSbnd{\typvar}{\typ} \\
                               && \vert & \ValSbnd{\expvar\TypAnn\typ}{\exp} \\
                               && \vert & \ModSbnd{\modvar}{\mod} \\
                               && \vert & \ModSbnd{\modvar\SigAnn\sigvar}{\mod} \\
                               && \vert & \FunctorSbnd{\modvar\SigAnn\sigvar}{\mod} \\
\end{longtable}
%%%%%%%%%%%%%%%%%%%%%%%%%%%%%%%%%%%%%%%%%%%%%%%%%%%%%%%%%%%%
% rules
\section{declarative statics}
    \subsection*{}
\section{contexts}
        \[
            \HoleCtx, ?;\ExpVarCtx, \expvar\TypAssump\typ; \TypVarCtx, \typvar\KndAssump\knd; \ModVarCtx, \modvar\SigAssump\sig; \SigVarCtx, \sigvar\SigKndAssump\sigknd
        \]
\section{statics}
    \subsection*{}
    \begin{minipage}{\textwidth}
        scratch \\
        \judgbox{\ConsistentSubkind{\HoleCtx;\TypVarCtx}{\knd[_1]}{\knd[_2]}}{$\knd[_1]$ is a consistent subkind of $\knd[_2]$}
        \begin{mathpar}
            \inferrule[KCSubsumption]{test}{test}
        \end{mathpar}
    \end{minipage}
    \begin{minipage}{\textwidth}
        \judgbox{\ConsistentSubSigKind{\HoleCtx;\TypVarCtx;\ModVarCtx;\SigVarCtx}{\sigknd[_1]}{\sigknd[_2]}}{$\sigknd[_1]$ is a consistent sub signature kind of $\sigknd[_2]$}
        \begin{mathpar}
            \inferrule[nameMe]{
            \exists\sdec[_x]\in\sdecs[_1]~st~\ConsistentSubSigKind{\HoleCtx;\TypVarCtx;\ModVarCtx;\SigVarCtx}{\SSigKind{\Sig{\sdec[_x]}}}{\SSigKind{\Sig{\sdec[_2]}}} \\
            \ConsistentSubSigKind{\HoleCtx;\TypVarCtx, \type{\HoleCtx;\TypVarCtx;\ModVarCtx;\SigVarCtx}{\sdec[_2]};\ModVarCtx, \submodule{\sdec[_2]};\SigVarCtx}{\Sig{\sdecs[_1]}}{\Sig{\sdecs[_2]}} \\
            }
            {\ConsistentSubSigKind{\HoleCtx;\TypVarCtx;\ModVarCtx;\SigVarCtx}{\SSigKind{\Sig{\sdec[_{11}], \sdec[_{12}], \sdecs[_{13}]~as~\sdecs[_1]}}}{\SSigKind{\Sig{\sdec[_2], \sdecs[_2]}}}}
            \and
            \inferrule[single]{
                \SubSdec{\HoleCtx;\TypVarCtx;\ModVarCtx;\SigVarCtx}{\sdec[_1]}{\sdec[_2]} \\
            }
            {\ConsistentSubSigKind{\HoleCtx;\TypVarCtx;\ModVarCtx;\SigVarCtx}{\SSigKind{\Sig{\sdec[_1]}}}{\SSigKind{\Sig{\sdec[_2]}}}}
            \and
            \inferrule[nil]{
            }
            {\ConsistentSubSigKind{\HoleCtx;\TypVarCtx;\ModVarCtx;\SigVarCtx}{\SSigKind{\Sig{\sdecs}}}{\SSigKind{\Sig{\cdot}}}}
            \and
            \inferrule[varprop]{
                \sigvar\SigKndAssump\sigknd\in\SigVarCtx
            }
            {\ConsistentSubSigKind{\HoleCtx;\TypVarCtx;\ModVarCtx;\SigVarCtx}{\SSigKind{\sigvar}}{\sigknd}}
            \and
            \inferrule[nameMe?delete?]{
                \sig[_1] \ne \sigvar \\
                \Ana{\HoleCtx;\TypVarCtx;\ModVarCtx;\SigVarCtx}{\sig[_1]}{\SSigKind{\sig[_2]}} \\
            }
            {\ConsistentSubSigKind{\HoleCtx;\TypVarCtx;\ModVarCtx;\SigVarCtx}{\SSigKind{\sig[_1]}}{\SSigKind{\sig[_2]}}}
            \and
            \inferrule[funct]{
                \ConsistentSubSigKind{\HoleCtx;\TypVarCtx;\ModVarCtx;\SigVarCtx}{\SSigKind{\sig[_{21}]}}{\SSigKind{\sig[_{11}]}} \\
                \ConsistentSubSigKind{\HoleCtx;\TypVarCtx;\ModVarCtx, \modvar\SigAssump\sig[_{11}];\SigVarCtx}{\SSigKind{\sig[_{12}]}}{\SSigKind{\sig[_{22}]}} \\
            }
            {\ConsistentSubSigKind{\HoleCtx;\TypVarCtx;\ModVarCtx;\SigVarCtx}{\SSigKind{\FunctorSig{\modvar\SigAnn\sig[_{11}]}{\sig[_{12}]}}}{\SSigKind{\FunctorSig{\modvar\SigAnn\sig[_{21}]}{\sig[_{22}]}}}}
            \and
            \inferrule[holes]{
                \Ana{\HoleCtx;\TypVarCtx;\ModVarCtx;\SigVarCtx}{\SigHole^{\hole}}{\sigknd} \\
            }
            {\ConsistentSubSigKind{\HoleCtx;\TypVarCtx;\ModVarCtx;\SigVarCtx}{\SSigKind{\SigHole^{\hole}}}{\sigknd}}
            \and
            \inferrule[neholes]{
                \Ana{\HoleCtx;\TypVarCtx;\ModVarCtx;\SigVarCtx}{\SigHole[s]^{\hole}}{\sigknd} \\
            }
            {\ConsistentSubSigKind{\HoleCtx;\TypVarCtx;\ModVarCtx;\SigVarCtx}{\SSigKind{\SigHole[s]^{\hole}}}{\sigknd}}
            \and
            \inferrule[CSubSigKindHoleL]{
            }
            {\ConsistentSubSigKind{\HoleCtx;\TypVarCtx;\ModVarCtx;\SigVarCtx}{\SigKHole}{\sigknd}}
            \and
            \inferrule[CSubSigKindHoleR]{
            }
            {\ConsistentSubSigKind{\HoleCtx;\TypVarCtx;\ModVarCtx;\SigVarCtx}{\sigknd}{\SigKHole}}
        \end{mathpar}
    \end{minipage}
    \begin{minipage}{\textwidth}
        \judgbox{\Syn{\HoleCtx;\TypVarCtx;\ModVarCtx;\SigVarCtx}{\sig}{\sigknd}}{$\sig$ synthesizes signature kind $\sigknd$}
        \begin{mathpar}
            \inferrule[SynSigKndVar]{
                \sigvar\SigKndAssump\sigknd\in\SigVarCtx \\
            }
            {\Syn{\HoleCtx;\TypVarCtx;\ModVarCtx;\SigVarCtx}{\sigvar}{\SSigKind{\sigvar}}}
            \and
            \inferrule[SynSigKndVarFail]{
                \sigvar\notin\dom{\SigVarCtx} \\
            }
            {\Syn{\HoleCtx;\TypVarCtx;\ModVarCtx;\SigVarCtx}{\sigvar}{\SigKHole}}
            \and
            \inferrule[]{
                \Sig{\sdecs} wellformed?
            }
            {\Syn{}{\Sig{\sdecs}}{\SSigKind{\Sig{\sdecs}}}}
            \and
            \inferrule[]{
                \Syn{\HoleCtx;\TypVarCtx;\ModVarCtx, \modvar\SigAssump\sig[_1];\SigVarCtx}{\sig[_2]}{\sigknd} \\
            }
            {\Syn{\HoleCtx;\TypVarCtx;\ModVarCtx;\SigVarCtx}{\FunctorSig{\modvar\SigAnn\sig[_1]}{\sig[_2]}}{\SSigKind{\FunctorSig{\modvar\SigAnn\sig[_1]}{\sig[_2]}}}}
            \and
            \inferrule[SynSigKndSigHole]{
                u\SigKndAssump\sigknd\in\HoleCtx \\
            }
            {\Syn{\HoleCtx;\TypVarCtx;\ModVarCtx;\SigVarCtx}{\SigHole^{\hole}}{\sigknd}}
            \and
            \inferrule[SynSigKndSigHole]{
                u\SigKndAssump\sigknd\in\HoleCtx \\
                \Syn{\HoleCtx;\TypVarCtx;\ModVarCtx;\SigVarCtx}{\sigvar}{\sigknd[_1]} \\
            }
            {\Syn{\HoleCtx;\TypVarCtx;\ModVarCtx;\SigVarCtx}{\SigHole[\sigvar]^{\hole}}{\sigknd}}
        \end{mathpar}
    \end{minipage}
    \begin{minipage}{\textwidth}
        \judgbox{\Ana{\HoleCtx;\TypVarCtx;\ModVarCtx;\SigVarCtx}{\sig}{\sigknd}}{$\sig$ analyzes against signature kind $\sigknd$}
        \begin{mathpar}
            \inferrule[Sub]{
                \Syn{\HoleCtx;\TypVarCtx;\ModVarCtx;\SigVarCtx}{\sig}{\sigknd[_1]} \\
                \ConsistentSubSigKind{\HoleCtx;\TypVarCtx;\ModVarCtx;\SigVarCtx}{\sigknd[_1]}{\sigknd}
            }
            {\Ana{\HoleCtx;\TypVarCtx;\ModVarCtx;\SigVarCtx}{\sig}{\sigknd}}
        \end{mathpar}
    \end{minipage}
    \begin{minipage}{\textwidth}
        \judgbox{\SubSdec{\HoleCtx;\TypVarCtx;\ModVarCtx;\SigVarCtx}{\sdec[_1]}{\sdec[_2]}}{$\sdec[_1]$ is a subsdec of $\sdec[_2]$}
        \begin{mathpar}
            \inferrule[singleType]{
            }
            {\SubSdec{\HoleCtx;\TypVarCtx;\ModVarCtx;\SigVarCtx}{\TransparentTypeSdec{\lab}{\typ}}{\OpaqueTypeSdec{\lab}}}
            \and
            \inferrule[singleType2]{
                \TypeEquiv{\HoleCtx;\TypVarCtx;\ModVarCtx;\SigVarCtx}{\typ[_1]}{\typ[_2]} \\
            }
            {\SubSdec{\HoleCtx;\TypVarCtx;\ModVarCtx;\SigVarCtx}{\TransparentTypeSdec{\lab}{\typ[_1]}}{\TransparentTypeSdec{\lab}{\typ[_2]}}}
            \and
            \inferrule[singleType3]{
            }
            {\SubSdec{\HoleCtx;\TypVarCtx;\ModVarCtx;\SigVarCtx}{\OpaqueTypeSdec{\lab}}{\OpaqueTypeSdec{\lab}}}
            \and
            \inferrule[singleVa]{
                \TypeEquiv{\HoleCtx;\TypVarCtx;\ModVarCtx;\SigVarCtx}{\typ[_1]}{\typ[_2]} \\
            }
            {\SubSdec{\HoleCtx;\TypVarCtx;\ModVarCtx;\SigVarCtx}{\ValSdec{\lab}{\typ[_1]}}{\ValSdec{\lab}{\typ[_2]}}}
            \and
            \inferrule[singleMod]{
                \Ana{\HoleCtx;\TypVarCtx;\ModVarCtx;\SigVarCtx}{\sig[_1]}{\SSigKind{\sig[_2]}} \\
            }
            {\SubSdec{\HoleCtx;\TypVarCtx;\ModVarCtx;\SigVarCtx}{\ModSdec{\lab}{\sig[_1]}}{\ModSdec{\lab}{\sig[_2]}}}
        \end{mathpar}
    \end{minipage}
\section{elab}
    \subsection*{}
    \begin{minipage}{\textwidth}
        \judgbox{\SynElab{\ExpVarCtx;\TypVarCtx;\ModVarCtx}{\uexp}{\typ}{\exp}{\HoleCtx}}{$\uexp$ synthesizes type $\typ$ and elaborates to $\exp$ with hole context $\HoleCtx$}
        \begin{mathpar}
            \inferrule[...]{ }{ }
            \and
            \inferrule[SynElabLetMod]{
                \SynElab{\ExpVarCtx;\TypVarCtx;\ModVarCtx}{\umod}{\sig}{\mod}{\HoleCtx_1} \\
                \SynElab{\ExpVarCtx;\TypVarCtx;\ModVarCtx, \modvar\SigAssump\sig}{\uexp}{\typ}{\exp}{\HoleCtx_2} \\
            }
            {\SynElab{\ExpVarCtx;\TypVarCtx;\ModVarCtx}{\uLetMod{\umodvar}{\umod}}{\typ}
                {\LetMod{\modvar}{\mod}}{\HoleCtx_1\union\HoleCtx_2}}
            \and
            \inferrule[SynElabLetModAnn]{
                \SynElab{\TypVarCtx;\ModVarCtx}{\usig}{\sigknd}
                    {\sig}{\HoleCtx_1} \\
                \AnaElab{\ExpVarCtx;\TypVarCtx;\ModVarCtx}{\umod}{\sig}
                    {\mod}{\HoleCtx_2} \\
                \SynElab{\ExpVarCtx;\TypVarCtx;\ModVarCtx, \modvar\SigAssump\sig}{\uexp}{\typ}
                    {\exp}{\HoleCtx_3} \\
            }
            {\SynElab{\ExpVarCtx;\TypVarCtx;\ModVarCtx}{\uLetMod{\umodvar\SigAnn\usig}{\umod}}{\typ}
                {\LetMod{\modvar\SigAnn\sig}{\mod}}{\HoleCtx_1\union\HoleCtx_2\union\HoleCtx_3}}
            \and
            \inferrule[SynElabModTermPrj]{
                \SynElab{\ExpVarCtx;\TypVarCtx;\ModVarCtx}{\umod}{\sig}
                    {\mod}{\HoleCtx} \\
                \Syn{\TypVarCtx;\ModVarCtx}{\sig}{\sigknd} \\
            }
            {\SynElab{\ExpVarCtx;\TypVarCtx;\ModVarCtx}{\uModTermPrj{\umod}{\lab}}{\typ}
                {\ModTermPrj{\mod}{\lab}}{\HoleCtx}}
        \end{mathpar}
    \end{minipage}
    \begin{minipage}{\textwidth}
        \judgbox{\SynElab{\TypVarCtx;\ModVarCtx}{\utyp}{\knd}{\typ}{\HoleCtx}}{$\utyp$ synthesizes kind $\knd$ and elaborates to $\typ$ with hole context $\HoleCtx$}
        \begin{mathpar}
            \inferrule[...]{ }{ }
            \and
            \inferrule[SynElabModTypPrj]{
                \SynElab{\TypVarCtx;\ModVarCtx}{\umodvar}{\sig}
                    {\modvar}{\HoleCtx} \\
                something \sig \knd \\
            }
            {\SynElab{\TypVarCtx;\ModVarCtx}{\ModTypPrj{\umodvar}{\lab}}{\knd}
                {\ModTypPrj{\modvar}{\lab}}{\HoleCtx}}
        \end{mathpar}
    \end{minipage}
    \begin{minipage}{\textwidth}
        \judgbox{\AnaElab{\TypVarCtx;\ModVarCtx}{\utyp}{\knd}{\typ}{\HoleCtx}}{$\utyp$ analyzes against kind $\knd$ and elaborates to $\typ$ with hole context $\HoleCtx$}
    \end{minipage}
    \begin{minipage}{\textwidth}
        \judgbox{\SynElab{\ExpVarCtx;\TypVarCtx;\ModVarCtx}{\umod}{\sig}{\mod}{\HoleCtx}}{$\umod$ synthesizes signature $\sig$ and elaborates to $\mod$ with hole context $\HoleCtx$}
        \begin{mathpar}
            \inferrule[SynElabModVar]{
                \umodvar\SigAssump\sig \in \ModVarCtx \\
            }
            {\SynElab{\ExpVarCtx;\TypVarCtx;\ModVarCtx}{\umodvar}{\sig}
                {\modvar}{\cdot}}
            \and
            \inferrule[SynElabModVarFail]{
                \umodvar \notin \dom{\ModVarCtx} \\
            }
            {\SynElab{\ExpVarCtx;\TypVarCtx;\ModVarCtx}{\umodvar}{\SigHole}
                {\ModHole[\modvar]^{\hole}}{u\SigAssump\SigHole}}
            \and
            \inferrule[SynElabConsStruct]{
                \SynElab{\ExpVarCtx;\TypVarCtx;\ModVarCtx}{\usbnd}{\sdec}
                    {\sbnd}{\HoleCtx_1} \\
                \SynElab{\ExpVarCtx, \val{\sdec};\TypVarCtx, \type{\HoleCtx_1;\TypVarCtx;\ModVarCtx;\SigVarCtx}{\sdec};\ModVarCtx, \submodule{\sdec}}{\Struct{\usbnds}}{\Sig{\sdecs}}
                    {\Struct{\sbnds}}{\HoleCtx_2} \\
            }
            {\SynElab{\ExpVarCtx;\TypVarCtx;\ModVarCtx}{\Struct{\usbnd, \usbnds}}{\Sig{\sdec, \sdecs}}
                {\Struct{\sbnd, \sbnds}}{\HoleCtx_1\union\HoleCtx_2}}
            \and
            \inferrule[SynElabNilStruct]{
            }
            {\SynElab{\ExpVarCtx;\TypVarCtx;\ModVarCtx}{\Struct{\cdot}}{\Sig{\cdot}}
                {\Struct{\cdot}}{\cdot}}
            \and
            \inferrule[SynElabEmptyModHole]{
            }
            {\SynElab{\ExpVarCtx;\TypVarCtx;\ModVarCtx}{\ModHole^{\hole}}{\SigHole}
                {\ModHole^{\hole}}{u\SigAssump\SigHole}}
            \and
            \inferrule[SynElabNonemptyModHole]{
            }
            {\SynElab{\ExpVarCtx;\TypVarCtx;\ModVarCtx}{\ModHole[\umodvar]^{\hole}}{\SigHole}
                {\ModHole[\modvar]^{\hole}}{u\SigAssump\SigHole}}
            \and
            \inferrule[functor stuff]{
            }
            { }
        \end{mathpar}
    \end{minipage}
    \begin{minipage}{\textwidth}
        \judgbox{\AnaElab{\ExpVarCtx;\TypVarCtx;\ModVarCtx}{\umod}{\sig}{\mod}{\HoleCtx}}{$\umod$ analyzes against signature $\sig$ and elaborates to $\mod$ with hole context $\HoleCtx$}
        \begin{mathpar}
            \inferrule[AnaElabModSubsumption]{
                \SynElab{\ExpVarCtx;\TypVarCtx;\ModVarCtx}{\umod}{\sig}
                    {\mod}{\HoleCtx} \\
            }
            {\AnaElab{\ExpVarCtx;\TypVarCtx;\ModVarCtx}{\umod}{\sig}
                {\mod}{\HoleCtx}}
        \end{mathpar}
    \end{minipage}
    \begin{minipage}{\textwidth}
        \judgbox{\SynElab{\ExpVarCtx;\TypVarCtx;\ModVarCtx}{\usbnd}{\sdec}{\sbnd}{\HoleCtx}}{$\usbnd$ synthesizes declaration $\sdec$ and elaborates to $\sbnd$ with hole context $\HoleCtx$}
        \begin{mathpar}
            \inferrule[SynElabTypeSbnd]{
                \SynElab{\TypVarCtx;\ModVarCtx}{\utyp}{\knd}
                    {\typ}{\HoleCtx} \\
            }
            {\SynElab{\ExpVarCtx;\TypVarCtx;\ModVarCtx}{\TypeSbnd{\utypvar}{\utyp}}{\TransparentTypeSdec{\typvar}{\typ}}
                {\TypeSbnd{\typvar}{\typ}}{\HoleCtx}}
            \and
            \inferrule[SynElabValSbnd]{
                \SynElab{\TypVarCtx;\ModVarCtx}{\utyp}{\knd}
                    {\typ}{\HoleCtx_1} \\
                \AnaElab{\ExpVarCtx;\TypVarCtx;\ModVarCtx}{\uexp}{\typ}
                    {\exp}{\HoleCtx_2} \\
            }
            {\SynElab{\ExpVarCtx;\TypVarCtx;\ModVarCtx}{\ValSbnd{\uexpvar\TypAnn\utyp}{\uexp}}{\ValSdec{\expvar}{\typ}}
                {\ValSbnd{\expvar\TypAnn\typ}{\exp}}{\HoleCtx_1\union\HoleCtx_2}}
            \and
            \inferrule[SynElabModSbnd]{
                \SynElab{\ExpVarCtx;\TypVarCtx;\ModVarCtx}{\umod}{\sig}
                    {\mod}{\HoleCtx} \\
            }
            {\SynElab{\ExpVarCtx;\TypVarCtx;\ModVarCtx}{\ModSbnd{\umodvar}{\umod}}{\ModSdec{\modvar}{\sig}}
            {\ModSbnd{\modvar\SigAnn\sig}{\mod}}{\HoleCtx}}
            \and
            \inferrule[SynElabModAnnSbnd]{
                \SynElab{\TypVarCtx;\ModVarCtx}{\usig}{\sigknd}
                    {\sig[_1]}{\HoleCtx_1} \\
                \SynElab{\ExpVarCtx;\TypVarCtx;\ModVarCtx}{\umod}{\sig[_2]}
                    {\mod}{\HoleCtx_2} \\
                \Ana{\TypVarCtx;\ModVarCtx;\SigVarCtx}{\sig[_2]}{\sigknd} \\
            }
            {\SynElab{\ExpVarCtx;\TypVarCtx;\ModVarCtx}{\ModSbnd{\umodvar\SigAnn\usig}{\umod}}{\ModSdec{\modvar}{\sig[_1]}}
            {\ModSbnd{\modvar\SigAnn\sig[_1]}{\mod}}{\HoleCtx_1\union\HoleCtx_2}}
        \end{mathpar}
    \end{minipage}
    \begin{minipage}{\textwidth}
        \judgbox{\AnaElab{\ExpVarCtx;\TypVarCtx;\ModVarCtx}{\usbnd}{\sdec}{\sbnd}{\HoleCtx}}{$\usbnd$ analyzes against declaration $\sdec$ and elaborates to $\sbnd$ with hole context $\HoleCtx$}
        \begin{mathpar}
            \inferrule[subsump]{
                \SynElab{\ExpVarCtx;\TypVarCtx;\ModVarCtx;l\SigVarCtx}{\usbnd}{\sdec[_1]}
                    {\sbnd}{\HoleCtx} \\
                \SubSdec{\HoleCtx;\TypVarCtx;\ModVarCtx;\SigVarCtx}{\sdec[_1]}{\sdec} \\
            }
            {\AnaElab{\ExpVarCtx;\TypVarCtx;\ModVarCtx;\SigVarCtx}{\usbnd}{\sdec}
                {\sbnd}{\HoleCtx}}
        \end{mathpar}
    \end{minipage}
    \begin{minipage}{\textwidth}
        \judgbox{\ElabDegenerate{\ExpVarCtx;\TypVarCtx;\ModVarCtx;\SigVarCtx}{\usdec}{\sdec}{\HoleCtx}}{$\usdec$ elaborates to $\sdec$ with hole context $\HoleCtx$}
        \begin{mathpar}
            \inferrule[opq]{
            }
            {\ElabDegenerate{\ExpVarCtx;\TypVarCtx;\ModVarCtx;\SigVarCtx}{\OpaqueTypeSdec{\lab}}
                {\OpaqueTypeSdec{\lab}}{\cdot}}
            \and
            \inferrule[trn]{
                \SynElab{\ExpVarCtx;\TypVarCtx;\ModVarCtx;\SigVarCtx}{\utyp}{\knd}
                    {\typ}{\HoleCtx}
            }
            {\ElabDegenerate{\ExpVarCtx;\TypVarCtx;\ModVarCtx;\SigVarCtx}{\TransparentTypeSdec{\lab}{\utyp}}
                {\TransparentTypeSdec{\lab}{\typ}}{\HoleCtx}}
            \and
            \inferrule[val]{
                \SynElab{\ExpVarCtx;\TypVarCtx;\ModVarCtx;\SigVarCtx}{\utyp}{\knd}
                    {\typ}{\HoleCtx}
            }
            {\ElabDegenerate{\ExpVarCtx;\TypVarCtx;\ModVarCtx;\SigVarCtx}{\ValSdec{\lab}{\utyp}}
                {\ValSdec{\lab}{\typ}}{\HoleCtx}}
            \and
            \inferrule[mod]{
                \SynElab{\ExpVarCtx;\TypVarCtx;\ModVarCtx;\SigVarCtx}{\usig}{\sigknd}
                    {\sig}{\HoleCtx}
            }
            {\ElabDegenerate{\ExpVarCtx;\TypVarCtx;\ModVarCtx;\SigVarCtx}{\ModSdec{\lab}{\usig}}
            {\ModSdec{\lab}{\sig}}{\HoleCtx}}
            \and
            \inferrule[we're going to need HOFunctors so we don't need to preclude users from typing a functor into a module and vice versa]{
            }
            {}
        \end{mathpar}
    \end{minipage}
    \begin{minipage}{\textwidth}
        \judgbox{\SynElab{\TypVarCtx;\ModVarCtx;\SigVarCtx}{\usig}{\sigknd}{\sig}{\HoleCtx}}{$\usig$ synthesizes signature kind $\sigknd$ and elaborates to $\sig$ with hole context $\HoleCtx$}
        \begin{mathpar}
            \inferrule[SynSigEmptyHole]{
            }
            {\SynElab{\TypVarCtx;\ModVarCtx;\SigVarCtx}{\SigHole^{\hole}}{\SigKHole}
                {\SigHole^{\hole}}{u\SigKndAssump\SigKHole}}
           \and
           \inferrule[SynSigNonEmptyHole]{
           }
           {}
        \end{mathpar}
    \end{minipage}
    \begin{minipage}{\textwidth}
        \judgbox{\AnaElab{\TypVarCtx;\ModVarCtx}{\usig}{\sigknd}{\sig}{\HoleCtx}}{$\usig$ analyzes against signature kind $\sigknd$ and elaborates to $\sig$ with hole context $\HoleCtx$}
    \end{minipage}
\subsection*{misc}
\begin{align*}
    \val{\sdec} &=
    \begin{cases}
        \lab\TypAssump\typ & \sdec = \ValSdec{\lab}{\typ} \\
        \cdot & \textrm{otherwise} \\
    \end{cases} \\
    \type{cntxts}{\sdec} &=
    \begin{cases}
        \lab\KndAssump\Type & \sdec = \OpaqueTypeSdec{\lab} \\
        \lab\KndAssump\knd & \sdec = \TransparentTypeSdec{\lab}{\typ} \\
                           & \textrm{where}~\Syn{cntxts}{\typ}{\knd}\\
        \cdot & \textrm{otherwise} \\
    \end{cases} \\
    \submodule{\sdec} &=
    \begin{cases}
        \lab\SigAssump\sig & \sdec = \ModSdec{\lab}{\sig} \\
        \cdot & \textrm{otherwise} \\
    \end{cases} \\
\end{align*}
\end{document}
