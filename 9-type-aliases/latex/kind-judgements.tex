\documentclass[12pt,letterpaper]{article}

\usepackage{mathpartir}
\usepackage{ latexsym }
\usepackage{stmaryrd}
\usepackage{ amssymb }
\usepackage{thmtools}

%% Joshua Dunfield macros
\def\OPTIONConf{1}%
\usepackage{joshuadunfield}


\newcommand{\elabAna}[5]{#1 \vdash #2 \Leftarrow #3 \leadsto #4 \dashv #5}
\newcommand{\elabSyn}[5]{#1 \vdash #2 \Rightarrow #3 \leadsto #4 \dashv #5}
\newcommand{\patMatch}[5]{#1 \vdash #2 : #3 \vartriangleright #4 \dashv #5}
\newcommand{\kconsubkind}[3]{#1 \vdash #2 \lesssim #3}
\newcommand{\kindAssign}[3]{#1 \vdash #2 : #3}
\newcommand{\kindSyn}[3]{#1 \vdash #2 \Rightarrow #3}
\newcommand{\kindAna}[3]{#1 \vdash #2 \Leftarrow #3}
\newcommand{\kformation}[2]{#1 \vdash #2 \textt{ kind} }
\newcommand{\kequiv}[3]{#1 \vdash #2 \equiv #3}
\newcommand{\kcequiv}[3]{#1 \vdash #2 \equiv #3}
\newcommand{\ksubkind}[3]{#1 \vdash #2 <:: #3}
\newcommand{\tyvarValid}[1]{#1 \textt{ valid} }
\newcommand{\tyvarNotValid}[1]{\lnot(#1 \textt{ }\textt{valid}) }
\newcommand{\dexpTypeAssign}[3]{#1 \vdash #2 : #3}
\newcommand{\kindLeqLattice}[3]{#1 \vdash #2 \leq #3}

\newcommand{\hPhi}{\Phi}
\newcommand{\hGamma}{\Gamma}
\newcommand{\EmptyhPhi}{\cdot}
\newcommand{\EmptyDelta}{\cdot}

\newcommand{\hexp}{e}
\newcommand{\dexp}{d}
\newcommand{\htau}{\hat{\tau}}
\newcommand{\hkappa}{\kappa}
\newcommand{\dtau}{\tau}
\newcommand{\hrho}{\rho}

\newcommand{\Ty}{\textt{Ty}}
\newcommand{\KHole}{\textt{KHole}}
\newcommand{\KSing}[2]{\textt{S}_{#1}\textt{(}#2{\textt{)}}}
\newcommand{\hlist}[1]{\textt{list(}#1\textt{)}}

\newcommand{\llparenthesiscolor}{\textcolor{violet}{\llparenthesis}}
\newcommand{\rrparenthesiscolor}{\textcolor{violet}{\rrparenthesis}}
\newcommand{\llparenthesiscolorc}{\textcolor{cyan}{\llparenthesis}}
\newcommand{\rrparenthesiscolorc}{\textcolor{cyan}{\rrparenthesis}}
\newcommand{\hthole}[1]{\llparenthesiscolor\rrparenthesiscolor^{#1}}
\newcommand{\hhole}[2]{\llparenthesiscolor#1\rrparenthesiscolor^{#2}}
\newcommand{\patehole}{\llparenthesiscolorc\rrparenthesiscolorc}
\newcommand{\pathole}[1]{\llparenthesiscolorc#1\rrparenthesiscolorc}

\newcommand{\Dbinding}[2]{#1 :: #2}
\newcommand{\domOf}[1]{\mathsf{dom}(#1)}

\newcommand{\htdefine}[3]{\textt{type }#1 \textt{ = } #2\textt{ in }#3}
\newcommand{\dtdefine}[4]{\textt{type }#1 \textt{ = } #2 : #3\textt{ in }#4}

\begin{document}

\begin{figure}[t]
	$\arraycolsep=4pt\begin{array}{rllllll}
			\mathsf{BinOp}               & \oplus & ::= &
			\textt{Product} ~\vert~ \textt{Sum} ~\vert~ \textt{Arrow}                                          \\
			\mathsf{Kind}                & \kappa & ::= &
			\Ty ~\vert~ \KHole ~\vert~ \KSing{\hkappa}{\tau}                                                   \\
			\mathsf{Constant Types}      & c      & ::= &
			\textt{Int} ~\vert~ \textt{Float} ~\vert~ \textt{Bool}                                             \\
			\mathsf{User HTyp}           & \htau  & ::= &
			c ~\vert~ \htau_1 \oplus \htau_2 ~\vert~ \hlist{\htau} ~\vert~ \hthole{u} ~\vert~ \hhole{\htau}{u} \\
			\mathsf{Internal HTyp}       & \dtau  & ::= &
			c ~\vert~ \dtau_1 \oplus \dtau_2 ~\vert~ \hlist{\dtau} ~\vert~ \hthole{u} ~\vert~ \hhole{\dtau}{u} \\
			\mathsf{Type Vars}           & t      &     &                                                      \\
			\mathsf{Type Pattern}        & \hrho  & ::= &
			t ~\vert~ \patehole ~\vert~ \pathole{t}                                                            \\
			\mathsf{User Expression}     & \hexp  & ::= &
			\htdefine{\hrho}{\htau}{\hexp} ~\vert~ elided                                                      \\
			\mathsf{Internal Expression} & \dtau  & ::= &
			\dtdefine{\hrho}{\dtau}{\hkappa}{\dexp} ~\vert~ elided                                             \\
		\end{array}$
\end{figure}

\begin{minipage}{\linewidth}
	\judgbox
	{\kconsubkind{\Delta;\hPhi}{\hkappa_1}{\hkappa_2}}
	{$\hkappa_1$ is a consistent subkind of $\hkappa_2$}
	\begin{mathpar}
		\inferrule[KCHoleL]{ }{
			\kconsubkind{\Delta;\hPhi}{\KHole}{\hkappa}
		}
		\and
		\inferrule[KCHoleR]{ }{
			\kconsubkind{\Delta;\hPhi}{\hkappa}{\KHole}
		}
		\and
		\inferrule[KCRespectEquiv]{
			\kequiv{\Delta;\hPhi}{\hkappa_1}{\hkappa_2} \\
		} {
			\kconsubkind{\Delta;\hPhi}{\hkappa_1}{\hkappa_2}
		}
		\and
		\inferrule[KCSubsumption]{
			\kindAna{\Delta;\hPhi}{\dtau}{\Ty}
		} {
			\kconsubkind{\Delta;\hPhi}{\KSing{\hkappa}{\dtau}}{\Ty}
		}
	\end{mathpar}
\end{minipage}
\\
\\


\begin{minipage}{\linewidth}
	\judgbox
	{\tyvarValid{t}}
	{$t$ is a valid type variable} \; \\
	$t$ is valid if it is not a builtin-type or keyword, begins with an alpha char or underscore, and only contains alphanumeric characters, underscores, and primes.
\end{minipage}
\\
\\


\begin{minipage}{\linewidth}
	\judgbox
	{\kformation{\Delta;\hPhi}{\hkappa}}
	{$\hkappa$ forms a kind} \;\\
	\begin{mathpar}
		\inferrule[KFTy] { }{ \kformation{\Delta;\hPhi}{\Ty} }
		\and
		\inferrule[KFHole] { }{ \kformation{\Delta;\hPhi}{\KHole} }
		\and
		\inferrule[KFSing]{
			\kindAna{\Delta;\hPhi}{\dtau}{\Ty}
		} {
			\kformation{\Delta;\hPhi}{\KSing{\hkappa}{\dtau}}
		}
	\end{mathpar}
\end{minipage}
\\
\\

\begin{minipage}{\linewidth}
	\judgbox
	{\kequiv{\Delta;\hPhi}{\hkappa_1}{\hkappa_2}}
	{$\hkappa_1$ is equivalent to $\hkappa_2$} \;\\
	\begin{mathpar}
		\inferrule[KERefl]{ } {
			\kequiv{\Delta;\hPhi}{\hkappa}{\hkappa}
		}
		\and
		\inferrule[KESymm]{
			\kequiv{\Delta;\hPhi}{\hkappa_1}{\hkappa_2}
		} {
			\kequiv{\Delta;\hPhi}{\hkappa_2}{\hkappa_1}
		}
		\and
		\inferrule[KETrans]{
			\kequiv{\Delta;\hPhi}{\hkappa_1}{\hkappa_2} \\
			\kequiv{\Delta;\hPhi}{\hkappa_2}{\hkappa_3}
		} {
			\kequiv{\Delta;\hPhi}{\hkappa_1}{\hkappa_3}
		}
		\and
		\inferrule[KESingEquiv]{
			\kcequiv{\Delta;\hPhi}{\dtau_1}{\dtau_2}
		} {
			\kequiv{\Delta;\hPhi}{\KSing{\hkappa_1}{\dtau_1}}{\KSing{\hkappa_2}{\dtau_2}}
		}
		\and
		\inferrule[KESingSing]{
			\kcequiv{\Delta;\hPhi}{\dtau_1}{\dtau_2}
		} {
			\kequiv{\Delta;\hPhi}{\KSing{\KSing{\hkappa_1}{\dtau_1}}{\dtau_3}}{\KSing{\hkappa_2}{\dtau_2}}
		}
	\end{mathpar}
\end{minipage}
\\
\\

\begin{minipage}{\linewidth}
	\judgbox
	{\kindSyn{\Delta;\hPhi}{\dtau}{\hkappa}}
	{$\dtau$ synthesizes kind $\hkappa$}
	\begin{mathpar}
		\inferrule[KSConst]{ }{
			\kindSyn{\Delta;\hPhi}{c}{\KSing{\hkappa}{c}}
		}
		\and
		\inferrule[KSVar]{
			t : \hkappa \in \hPhi
		} {
			\kindSyn{\Delta;\hPhi}{t}{\KSing{\hkappa}{t}}
		}
		\and
		\inferrule[KSUVar]{
			t \not \in \domOf{\hPhi}
		} {
			\kindSyn{\Delta;\hPhi}{t}{\KHole}
		}
		\and
		\inferrule[KSBinOp]{
			\kindAna{\Delta;\hPhi}{\dtau_1}{\KSing{\hkappa}{\dtau_1}} \\
			\kindAna{\Delta;\hPhi}{\dtau_2}{\KSing{\hkappa}{\dtau_2}}
		} {
			\kindSyn{\Delta;\hPhi}{\dtau_1 \oplus \dtau_2}{\KSing{\hkappa}{\dtau_1 \oplus \dtau_2}}
		}
		\and
		\inferrule[KSList]{
			\kindAna{\Delta;\hPhi}{\dtau}{\KSing{\hkappa}{\dtau}}
		} {
			\kindSyn{\Delta;\hPhi}{\hlist{\dtau}}{\KSing{\hkappa}{\hlist{\dtau}}}
		}
		\and
		\inferrule[KSEHole]{
			\Dbinding{u}{\hkappa} \in \Delta
		} {
			\kindSyn{\Delta;\hPhi}{\hthole{u}}{\hkappa}
		}
		\and
		\inferrule[KSNEHole]{
			\Dbinding{u}{\hkappa} \in \Delta \\
			\kindSyn{\Delta;\hPhi}{\dtau}{\hkappa'}
		} {
			\kindSyn{\Delta;\hPhi}{\hhole{\dtau}{u}}{\hkappa}
		}


	\end{mathpar}
\end{minipage}
\\
\\



\begin{minipage}{\linewidth}
	\judgbox
	{\kindAna{\Delta;\hPhi}{\dtau}{\hkappa}}
	{$\dtau$ analyzes against kind $\hkappa$}
	\begin{mathpar}
		\inferrule[KAASubsume]{
			\kindSyn{\hPhi}{\dtau}{\hkappa'} \\
			\kconsubkind{\Delta;\hPhi}{\hkappa'}{\hkappa}
		}{
			\kindAna{\Delta;\hPhi}{\dtau}{\hkappa}
		}


	\end{mathpar}
\end{minipage}
\\
\\



\begin{minipage}{\linewidth}
	\judgbox
	{\kcequiv{\Delta;\hPhi}{\dtau_1}{\dtau_2}}
	{$\dtau_1$ is equivalent to $\dtau_2$} \;\\
	\begin{mathpar}
		\inferrule[KCESymm]{
			\kcequiv{\Delta;\hPhi}{\dtau_1}{\dtau_2}
		} {
			\kcequiv{\Delta;\hPhi}{\dtau_2}{\dtau_1}
		}
		\and
		\inferrule[KCETrans]{
			\kcequiv{\Delta;\hPhi}{\dtau_1}{\dtau_2} \\
			\kcequiv{\Delta;\hPhi}{\dtau_2}{\dtau_3}
		} {
			\kcequiv{\Delta;\hPhi}{\dtau_1}{\dtau_3}
		}
		\and
		\inferrule[KCESingEquiv]{
			\kindAna{\Delta;\hPhi}{\dtau_1}{\KSing{\hkappa}{\dtau_2}}
		} {
			\kcequiv{\Delta;\hPhi}{\dtau_1}{\dtau_2}
		}
		\and
		\inferrule[KCEConst]{ }{
			\kcequiv{\Delta;\hPhi}{c}{c}
		}
		\and
		\inferrule[KCEVar]{
			t : \hkappa \in \hPhi
		} {
			\kcequiv{\Delta;\hPhi}{t}{t}
		}
		\and
		\inferrule[KCEBinOp]{
			\kcequiv{\Delta;\hPhi}{\dtau_1}{\dtau_2} \\
			\kcequiv{\Delta;\hPhi}{\dtau_3}{\dtau_4}
		} {
			\kcequiv{\Delta;\hPhi}{\dtau_1 \oplus \dtau_3}{\dtau_2 \oplus \dtau_4}
		}
		\and
		\inferrule[KCEList]{
			\kcequiv{\Delta;\hPhi}{\dtau_1}{\dtau_2}
		} {
			\kcequiv{\Delta;\hPhi}{\hlist{\dtau_1}}{\hlist{\dtau_2}}
		}
		\and
		\inferrule[KCEEHole]{
			\Dbinding{u}{\hkappa} \in \Delta
		} {
			\kcequiv{\Delta;\hPhi}{\hthole{u}}{\hthole{u}}
		}
		\and
		\inferrule[KCENEHole]{
			\Dbinding{u}{\hkappa} \in \Delta \\
			\kindAna{\Delta;\hPhi}{\dtau}{\hkappa'}
		} {
			\kcequiv{\Delta;\hPhi}{\hhole{\dtau}{u}}{\hhole{\dtau}{u}}
		}
	\end{mathpar}
\end{minipage}
\\
\\

\begin{minipage}{\linewidth}
	\judgbox
	{\elabSyn{\hPhi}{\htau}{\hkappa}{\dtau}{\Delta}}
	{$\htau$ synthesizes kind $\hkappa$ and elaborates to $\dtau$}
	\begin{mathpar}
		\inferrule[TElabSConst]{ }{
			\elabSyn{\hPhi}{c}{\KSing{\hkappa}{c}}{c}{\EmptyDelta}
		}
		\and
		\inferrule[TElabSBinOp]{
			\elabAna{\hPhi}{\htau_1}{\Ty}{\dtau_1}{\Delta_1} \\
			\elabAna{\hPhi}{\htau_2}{\Ty}{\dtau_2}{\Delta_2}
		}{
			\elabSyn{\hPhi}{\htau_1 \oplus \htau_2}{\KSing{\hkappa}{\dtau_1 \oplus \dtau_2}}{\dtau_1 \oplus \dtau_2}{\Delta_1 \cup \Delta_2}
		}
		\and
		\inferrule[TElabSList]{
			\elabAna{\hPhi}{\htau}{\Ty}{\dtau}{\Delta}
		} {
			\elabSyn{\hPhi}{\hlist{\htau}}{\KSing{\hkappa}{\hlist{\dtau}}}{\hlist{\dtau}}{\Delta}
		}
		\and
		\inferrule[TElabSVar]{
			t : \hkappa \in \hPhi
		} {
			\elabSyn{\hPhi}{t}{\KSing{\hkappa}{t}}{t}{\EmptyDelta}
		}
		\and
		\inferrule[TElabSUVar]{
			t \not\in \domOf{\hPhi}
		} {
			\elabSyn{\hPhi}{t}{\KHole}{\hhole{t}{u}}{\Dbinding{u}{\KHole} }
		}
		\and
		\inferrule[TElabSHole]{ } {
			\elabSyn{\hPhi}{\hthole{u}}{\KHole}{\hthole{u}}{\Dbinding{u}{\KHole}}
		}
		\and
		\inferrule[TElabSNEHole]{
			\elabSyn{\hPhi}{\htau}{\hkappa}{\dtau}{\Delta}
		} {
			\elabSyn{\hPhi}{\hhole{\htau}{u}}{\KHole}{\hhole{\dtau}{u}}{\Delta,\Dbinding{u}{\KHole}}
		}
	\end{mathpar}
\end{minipage}
\\
\\

\begin{minipage}{\linewidth}
	\judgbox
	{\elabAna{\hPhi}{\htau}{\hkappa}{\dtau}{\Delta}}
	{$\htau$ analyzes against kind $\hkappa_1$ and elaborates to $\dtau$}
	\begin{mathpar}
		\inferrule[TElabASubsume]{
			\htau \neq \hthole{u} \\
			\htau \neq \hhole{\htau'}{u} \\
			\elabSyn{\hPhi}{\htau}{\hkappa'}{\dtau}{\Delta} \\
			\kconsubkind{\Delta;\hPhi}{\hkappa'}{\hkappa}
		}{
			\elabAna{\hPhi}{\htau}{\hkappa}{\dtau}{\Delta}
		}
		\and
		\inferrule[TElabAEHole]{ } {
			\elabAna{\hPhi}{\hthole{u}}{\hkappa}{\hthole{u}}{\Dbinding{u}{\kappa}}
		}
		\and
		\inferrule[TElabANEHole]{
			\elabSyn{\hPhi}{\htau}{\hkappa'}{\dtau}{\Delta}
		} {
			\elabAna{\hPhi}{\hhole{\htau}{u}}{\hkappa}{\hhole{\dtau}{u}}{\Delta,\Dbinding{u}{\kappa}}
		}

	\end{mathpar}
\end{minipage}
\\
\\

\begin{minipage}{\linewidth}
	\judgbox
	{\patMatch{\hPhi_1}{\dtau}{\hkappa}{\hrho}{\hPhi_2}}
	{$\hrho$ matches against $\dtau : \hkappa$ extending $\hPhi$ if necessary}
	\begin{mathpar}
		\inferrule[RESVar]{
			\tyvarValid{t}
		}{
			\patMatch{\hPhi}{\dtau}{\hkappa}{t}{\hPhi,t :: \hkappa}
		}
		\and
		\inferrule[RESEHole]{ }{
			\patMatch{\hPhi}{\dtau}{\hkappa}{\patehole}{\hPhi}
		}
		\and
		\inferrule[RESVarHole]{
			\tyvarNotValid{t}
		}{
			\patMatch{\hPhi}{\dtau}{\hkappa}{\pathole{t}}{\hPhi}
		}
	\end{mathpar}
\end{minipage}
\\
\\

\begin{minipage}{\linewidth}
	\judgbox
	{\elabSyn{\hGamma;\hPhi}{\hexp}{\htau}{\dexp}{\Delta}}
	{$\hexp$ synthesizes type $\dtau$ and elaborates to $\dexp$}
	\begin{mathpar}
		\inferrule[ESDefine]{
			\elabSyn{\hPhi_1}{\htau}{\hkappa}{\dtau}{\Delta_1} \\
			\patMatch{\hPhi_1}{\dtau}{\hkappa}{\hrho}{\hPhi_2} \\
			\elabSyn{\hGamma;\hPhi_2}{\hexp}{\dtau_1}{\dexp}{\Delta_2}
		}{
			\elabSyn{\hGamma;\hPhi_1}{\htdefine{\hrho}{\htau}{\hexp}}{\dtau_1}{\dtdefine{\hrho}{\dtau}{\hkappa}{\dexp}}{\Delta_1 \cup \Delta_2}
		}


	\end{mathpar}
\end{minipage}
\\
\\

\begin{minipage}{\linewidth}
	\judgbox
	{\dexpTypeAssign{\Delta;\hGamma;\hPhi}{\dexp}{\dtau}}
	{$\dexp$ is assigned type $\dtau$}
	\begin{mathpar}
		\inferrule[DEDefine]{
			\patMatch{\hPhi_1}{\dtau_1}{\hkappa}{\hrho}{\hPhi_2} \\
			\dexpTypeAssign{\Delta;\hGamma;\hPhi_2}{\dexp}{\dtau_2}
		}{
			\dexpTypeAssign{\Delta;\hGamma;\hPhi_1}{\dtdefine{\hrho}{\dtau_1}{\hkappa}{\dexp}}{\dtau_2}
		}


	\end{mathpar}
\end{minipage}
\\
\\

\newtheorem{thm}{Theorem}
\begin{thm}[Well-Kinded Elaboration]\ \\
	(1) If $\elabSyn{\hPhi}{\htau}{\hkappa}{\dtau}{\Delta}$ then $\kindSyn{\Delta;\hPhi}{\dtau}{\hkappa}$ \\
	(2) If $\elabAna{\hPhi}{\htau}{\hkappa}{\dtau}{\Delta}$ then $\kindAna{\Delta;\hPhi}{\dtau}{\hkappa}$
\end{thm}
\noindent
This is like the Typed Elaboration theorem in the POPL19 paper.

\begin{thm}[Elaborability]\ \\
	(1) $\exists \Delta$ s.t. if $\kindSyn{\Delta;\hPhi}{\dtau}{\hkappa}$ then $\exists \htau$ such that $\elabSyn{\hPhi}{\htau}{\hkappa}{\dtau}{\Delta}$ \\
	(2) $\exists \Delta$ s.t. if $\kindAna{\Delta;\hPhi}{\dtau}{\hkappa}$ then $\exists \htau$ such that $\elabAna{\hPhi}{\htau}{\hkappa}{\dtau}{\Delta}$
\end{thm}
\noindent
This is similar but a little different from Elaborability theorem in the POPL19 paper. Choose the $\Delta$ that is emitted from elaboration and then there's an $\htau$ that elaborates to any of the $\dtau$ forms. Elaborability and Well-Kinded Elaboration implies we can just rely on the elaboration forms for the premises of any rules that demand kind synthesis/analysis.

\begin{thm}[Type Elaboration Unicity]\ \\
	(1) If $\elabSyn{\hPhi}{\htau}{\hkappa_1}{\dtau_1}{\Delta_1}$ and $\elabSyn{\hPhi}{\htau}{\hkappa_2}{\dtau_2}{\Delta_2}$ then $\hkappa_1 = \hkappa_2$, $\dtau_1 = \dtau_2$, $\Delta_1 = \Delta_2$ \\
	(2) If $\elabAna{\hPhi}{\htau}{\hkappa}{\dtau_1}{\Delta_1}$ and $\elabAna{\hPhi}{\htau}{\hkappa}{\dtau_2}{\Delta_2}$ then $\dtau_1 = \dtau_2$, $\Delta_1 = \Delta_2$
\end{thm}
\noindent
This is like the Elaboration Unicity theorem in the POPL19 paper.


\begin{thm}[Kind Synthesis Precision]\ \\
	If $\kindSyn{\Delta;\hPhi}{\dtau}{\hkappa_1}$ and $\kindAna{\Delta;\hPhi}{\dtau}{\hkappa_2}$ then $\kconsubkind{\Delta;\hPhi}{\hkappa_1}{\hkappa_2}$
\end{thm}

\noindent
Kind Synthesis Precision says that synthesis finds the most precise kappa possible for a given input type. This is somewhat trivial, but interesting to note because it means we can expect singletons wherever possible.




\end{document}
