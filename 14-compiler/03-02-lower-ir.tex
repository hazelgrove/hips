\documentclass[index.tex]{subfiles}

\begin{document}
\newcommand{\SyWrap}{\ensuremath{\textsf{\textbf{{wrap}}}}}
\newcommand{\SyCaseComplete}{\ensuremath{\textsf{\textbf{casecomplete}}}}

\newcommand{\NKMName}{\ensuremath{\textsf{Completeness}}}
\newcommand{\NKMV}{\ensuremath{k}}
\newcommand{\KMName}{\ensuremath{\textsf{}}}
\newcommand{\KMV}{\ensuremath{\overline{k}}}
\newcommand{\TCMName}{\ensuremath{\textsf{Type}}}
\newcommand{\TCMV}{\ensuremath{\tau}}
\newcommand{\TMName}{\ensuremath{\textsf{}}}
\newcommand{\TMV}{\ensuremath{\overline{\tau}}}
\newcommand{\CMName}{\ensuremath{\textsf{Composite}}}
\newcommand{\CMV}{\ensuremath{c}}
\newcommand{\IMName}{\ensuremath{\textsf{Immediate}}}
\newcommand{\IMV}{\ensuremath{i}}
\newcommand{\HIMV}{\ensuremath{u}}
\newcommand{\HEMV}{\ensuremath{\SyHoleEnv}}

\newcommand{\KNC}{\ensuremath{\pie{0.3ex}{360}}}
\newcommand{\KNI}{\ensuremath{\pie{0.3ex}{0}}}
\newcommand{\KII}{\ensuremath{\pie{0.3ex}{180}}}

\newcommand{\TCHole}{\ensuremath{\SyEHole{}{}}}
\newcommand{\TCInt}{\ensuremath{\SyTInt}}
\newcommand{\TCFloat}{\ensuremath{\SyTFloat}}
\newcommand{\TCBool}{\ensuremath{\SyTBool}}
\newcommand{\TCPair}[2]{\ensuremath{#1 \SyProd #2}}
\newcommand{\TCFun}[2]{\ensuremath{#1 \SyArrow #2}}

\newcommand{\TIntNC}{\ensuremath{\TMk{\TCInt}{\KNC}}}
\newcommand{\TIntNI}{\ensuremath{\TMk{\TCInt}{\KNI}}}
\newcommand{\TFloatNC}{\ensuremath{\TMk{\TCFloat}{\KNC}}}
\newcommand{\TFloatNI}{\ensuremath{\TMk{\TCFloat}{\KNI}}}
\newcommand{\TBoolNC}{\ensuremath{\TMk{\TCBool}{\KNC}}}
\newcommand{\TBoolNI}{\ensuremath{\TMk{\TCBool}{\KNI}}}
\newcommand{\TPairNC}[2]{\ensuremath{\TMk{\TCPair{#1}{#2}}{\KNC}}}
\newcommand{\TPairNI}[2]{\ensuremath{\TMk{\TCPair{#1}{#2}}{\KNI}}}
\newcommand{\TFunNC}[2]{\ensuremath{\TMk{(\TCFun{#1}{#2})}{\KNC}}}
\newcommand{\TFunNI}[2]{\ensuremath{\TMk{(\TCFun{#1}{#2})}{\KNI}}}
\newcommand{\TMk}[2]{\ensuremath{#1[#2]}}

\newcommand{\IIntMV}{\ensuremath{\underline{n}}}
\newcommand{\IFloatMV}{\ensuremath{\underline{f}}}
\newcommand{\IBoolMV}{\ensuremath{\underline{b}}}
\newcommand{\ITrue}{\ensuremath{\SyTrue}}
\newcommand{\IFalse}{\ensuremath{\SyFalse}}
\newcommand{\IEHole}[2]{\ensuremath{\SyEHole{#1}{#2}}}

\newcommand{\CPlus}[2]{\ensuremath{#1 \SyPlus #2}}
\newcommand{\CTimes}[2]{\ensuremath{#1 \SyTimes #2}}
\newcommand{\CFPlus}[2]{\ensuremath{#1 \SyFPlus #2}}
\newcommand{\CFTimes}[2]{\ensuremath{#1 \SyFTimes #2}}

\newcommand{\CAnd}[2]{\ensuremath{#1 \SyAnd #2}}
\newcommand{\COr}[2]{\ensuremath{#1 \SyOr #2}}

\newcommand{\CLam}[3]{\ensuremath{\SyFun #1 : #2 \SyDot~ #3}}
\newcommand{\CAp}[2]{\ensuremath{#1 ~#2}}

\newcommand{\CLet}[2]{\ensuremath{\SyLet~ #1 = #2}}
\newcommand{\CIn}{\ensuremath{~\SyIn~}}
\newcommand{\CInWith}[1]{\ensuremath{~\SyIn~ #1}}
\newcommand{\CReturn}[1]{\ensuremath{\SyReturn~ #1}}

\newcommand{\CPair}[2]{\ensuremath{(#1, #2)}}
\newcommand{\CProjL}[1]{\ensuremath{\SyProjL~ #1}}
\newcommand{\CProjR}[1]{\ensuremath{\SyProjR~ #1}}

\newcommand{\CWrapIntoNI}[1]{\ensuremath{\SyWrap^{\KNI}~ #1}}
\newcommand{\CWrapIntoII}[1]{\ensuremath{\SyWrap^{\KII}~ #1}}
\newcommand{\CEmbed}[1]{\ensuremath{\SyEmbed~ #1}}
\newcommand{\CProj}[2]{\ensuremath{\SyProj[#2]~ #1}}
\newcommand{\CCaseCompleteWith}[1]{\ensuremath{\SyCaseComplete~ #1 ~\SyWith}}
\newcommand{\CCaseCompleteBr}[2]{\ensuremath{\SyBar~ #1 \SyArrow #2}}

\subsection{Lower intermediate representation (LIR)}
\label{sec:lir}

\begin{note}
  The semantics of this representation are not totally thought out; wouldn't be surprised if there's
  some big inconsistency in this approach\ldots
\end{note}

In the lower intermediate representation (LIR), whose syntax is partially given in
\cref{fig:lir-syntax}, completeness (see \Cref{sec:lir-completeness}) is made explicit in the type
system (see \Cref{fig:lir-ta-imm,fig:lir-ta-comp}).

\begin{figure}[htb!]
  \[\begin{array}{rccl}
    \NKMName & \NKMV & \Coloneqq & \KNC \mid \KNI \\
    \KMName  & \KMV  & \Coloneqq & \NKMV \mid \KII \\
    \TCMName    & \TCMV    & \Coloneqq & \TCHole \mid \TCInt \mid \TCFloat \mid \TCBool 
                                                   \mid \TCPair{\TCMV}{\TCMV} 
                                                   \mid \TCFun{\TMk{\TCMV}{\NKMV}}{\TMV} \\
    \TMName       & \TMV       & \Coloneqq & \TMk{\TCMV}{\KMV} \\
    \IMName       & \IMV       & \Coloneqq & x \mid \IEHole{\HIMV}{\HEMV}
                                                   \mid \IIntMV \mid \IFloatMV \mid \IBoolMV 
                                                   \mid \CLam{x}{\TMk{\TCMV}{\NKMV}}{\CMV} \\
    \CMName      & \CMV      & \Coloneqq & \IMV \\
                   &                   & \mid         & \CPlus{\IMV}{\IMV} 
                                                   \mid \CTimes{\IMV}{\IMV}
                                                   \mid \CFPlus{\IMV}{\IMV} 
                                                   \mid \CFTimes{\IMV}{\IMV}
                                                   \mid \CAnd{\IMV}{\IMV}
                                                   \mid \COr{\IMV}{\IMV} \\
                   &                   & \mid         & \CAp{\IMV}{\IMV} \\
                   &                   & \mid         & \CPair{\IMV}{\IMV}
                                                   \mid \CProjL{\IMV}
                                                   \mid \CProjR{\IMV} \\
                   &                   & \mid         & \CLet{x}{\CMV} \CInWith{\CMV} \\
                   &                   & \mid         & \CWrapIntoNI{\IMV}
                                                   \mid \CWrapIntoII{\IMV} \\
                   &                   & \mid         & \CEmbed{\IMV}
                                                   \mid \CProj{\IMV}{\TCMV} \\
                   &                   & \mid         & \CCaseCompleteWith{\IMV}
                                                    ~\CCaseCompleteBr{x}{\CMV}
                                                    ~\CCaseCompleteBr{x'}{\CMV}
  \end{array}\]
  %
  \caption{LIR syntax}
  \label{fig:lir-syntax}
\end{figure}

\subsubsection{Completeness}
\label{sec:lir-completeness}
Completeness (see \Cref{completeness}) is denoted by the metavariable $\KMV$, with
``necessarily'' cases given by $\NKMV$: $\KNC =$ necessary completeness, $\KNI =$
necessary incompleteness, and $\KII =$ indeterminate incompleteness. They are encoded in types,
which are given by the metavariable $\TMV$. 
%
A type $\TMk{\TCMV}{\KMV}$ is thus a ``concrete'' type $\TCMV$ (need a
better name for this) parameterized by completeness $\KMV$.

\subsubsection{Immediates}
\label{sec:lir-immediates}
Similar to the high intermediate representation, a distinction is made between \emph{immediates}
(given by $\IMV$) and \emph{composites} (computations, given by $\CMV$). Immediates
are either literals or variables.

\begin{figure}[htb!]
  \judgbox{\ctxAssignType{\andCtx{\ctx}{\holeCtx}}{\IMV}{\TMV}}
    $\IMV$ is assigned type $\TMV$ \\
  
  \begin{mathpar}
    \judgment{
      \holeHasTypeCtx{\holeCtx}{\HIMV}{\TCMV}{\ctx'} \\
      \ctxAssignType{\andCtx{\ctx}{\holeCtx}}{\sigma}{\ctx'}
    }{
      \ctxAssignType{\andCtx{\ctx}{\holeCtx}}{\IEHole{\HIMV}{\sigma}}{\TMk{\TCMV}{\KNI}}
    }{TAEHole} \and

    \judgment{
      \inCtx{\ctx}{x}{\TMV}
    }{
      \ctxAssignType{\andCtx{\ctx}{\holeCtx}}{x}{\TMV}
    }{TAVar} \\

    \judgment{ }{
      \ctxAssignType{\andCtx{\ctx}{\holeCtx}}{\IIntMV}{\TIntNC}
    }{TAIntLit} \and

    \judgment{ }{
      \ctxAssignType{\andCtx{\ctx}{\holeCtx}}{\IIntMV}{\TFloatNC}
    }{TAFloatLit} \and

    \judgment{ }{
      \ctxAssignType{\andCtx{\ctx}{\holeCtx}}{\IBoolMV}{\TBoolNC}
    }{TABoolLit} \\

    \judgment{
      \ctxAssignType{\andCtx{\extendCtx{\ctx}{x}{\TMk{\TCMV}{\NKMV}}}{\holeCtx}}{\CMV}{\TMV}
    }{
      \ctxAssignType{\andCtx{\ctx}{\holeCtx}}{\CLam{x}{\TMk{\TCMV}{\NKMV}}{\CMV}}{\TFunNC{\TMk{\TCMV}{\NKMV}}{\TMV}}
    }{TALam}
  \end{mathpar}
  %
  \caption{Type assignment for LIR immediates}
  \label{fig:lir-ta-imm}
\end{figure}

\Cref{fig:lir-ta-imm} gives the type assignment rules for immediates, which are quite
straightforward. The rule for holes $\IEHole{\HIMV}{\HEMV}$ is the same as
that of the internal \Hazel{} language.

\subsubsection{Composites}
\label{sec:lir-composites}
See \cref{fig:lir-syntax-comp} for an brief overview of the composite forms. Note the operations that
are polymorphic over necessary completeness (e.g. integer addition).

\begin{table}
  \begin{center}
    \begin{tabular}{rl}
      \textbf{Form} & \textbf{Description} \\

      $\IMV$ 
        & immediates are also composites \\
      
      $\CPlus{\IMV}{\IMV}$, $\CTimes{\IMV}{\IMV}$
        & addition, multiplication of two $\TCInt$ immediates \\

      $\CFPlus{\IMV}{\IMV}$, $\CFTimes{\IMV}{\IMV}$
        & addition, multiplication of two $\TCFloat$ immediates \\

      $\CAnd{\IMV}{\IMV}$, $\COr{\IMV}{\IMV}$
        & logical and, logical or of two $\TCBool$ immediates \\

      $\CPair{\IMV}{\IMV}$
        & pair construction \\

      $\CProjL{\IMV}$, $\CProjR{\IMV}$
        & left and right pair projection \\

      $\CAp{\IMV}{\IMV}$
        & function application \\

      $\CWrapIntoNI{\IMV}$
        & wrapping of a value into an indet. result \\

      $\CWrapIntoII{\IMV}$
        & wrapping of an immediate into a indeterminately incomplete
        result \\

      $\CEmbed{\IMV}$
        & embedding of an immediate into a cast proxy \\

      $\CProj{\TCMV}{\IMV}$
        & projection of a proxy to a target type $\TCMV$; dynamic type check and may fail \\

      $\CLet{x}{\CMV} \CIn \CMV$
        & variable binding \\

      $\CCaseCompleteWith{\IMV} ~\cdots$
        & dynamic check; first branch taken if value, second branch if indet. result
    \end{tabular}
  \end{center}
  %
  \caption{Overview of LIR composite forms}
  \label{fig:lir-syntax-comp}
\end{table}

The syntax and type assignment rules (see \Cref{fig:lir-ta-comp}) enforce serious computations to be
on either necessarily complete or necessarily incomplete immediates; indeterminately incomplete
results may be introduced only either by wrapping (\textsc{\small TAWrapIntoII}) or by projecting a
cast proxy (\textsc{\small TAProj}), and eliminated only by $\SyCaseComplete$ (\textsc{\small
TACaseComplete}).

\begin{figure}[htb!]
  \judgbox{\ctxAssignType{\andCtx{\ctx}{\holeCtx}}{\CMV}{\TMV}}
    $\CMV$ is assigned type $\TMV$ \\

  Integers:
  \begin{mathpar}
    \judgment{
      \ctxAssignType{\andCtx{\ctx}{\holeCtx}}{\IMV_1}{\TMk{\TCInt}{\NKMV}} \\
      \ctxAssignType{\andCtx{\ctx}{\holeCtx}}{\IMV_2}{\TMk{\TCInt}{\NKMV}}
    }{
      \ctxAssignType{\andCtx{\ctx}{\holeCtx}}{\CPlus{\IMV_1}{\IMV_2}}{\TMk{\TCInt}{\NKMV}}
    }{TAPlus} \and

    \judgment{
      \ctxAssignType{\andCtx{\ctx}{\holeCtx}}{\IMV_1}{\TMk{\TCInt}{\NKMV}} \\
      \ctxAssignType{\andCtx{\ctx}{\holeCtx}}{\IMV_2}{\TMk{\TCInt}{\NKMV}}
    }{
      \ctxAssignType{\andCtx{\ctx}{\holeCtx}}{\CTimes{\IMV_1}{\IMV_2}}{\TMk{\TCInt}{\NKMV}}
    }{TATimes}
  \end{mathpar} \\
  %
  Floats:
  \begin{mathpar}
    \judgment{
      \ctxAssignType{\andCtx{\ctx}{\holeCtx}}{\IMV_1}{\TMk{\TCFloat}{\NKMV}} \\
      \ctxAssignType{\andCtx{\ctx}{\holeCtx}}{\IMV_2}{\TMk{\TCFloat}{\NKMV}}
    }{
      \ctxAssignType{\andCtx{\ctx}{\holeCtx}}{\CFPlus{\IMV_1}{\IMV_2}}{\TMk{\TCFloat}{\NKMV}}
    }{TAFPlus} \and

    \judgment{
      \ctxAssignType{\andCtx{\ctx}{\holeCtx}}{\IMV_1}{\TMk{\TCFloat}{\NKMV}} \\
      \ctxAssignType{\andCtx{\ctx}{\holeCtx}}{\IMV_2}{\TMk{\TCFloat}{\NKMV}}
    }{
      \ctxAssignType{\andCtx{\ctx}{\holeCtx}}{\CFTimes{\IMV_1}{\IMV_2}}{\TMk{\TCFloat}{\NKMV}}
    }{TAFTimes}
  \end{mathpar} \\
  %
  Booleans:
  \begin{mathpar}
    \judgment{
      \ctxAssignType{\andCtx{\ctx}{\holeCtx}}{\IMV_1}{\TMk{\TCBool}{\NKMV}} \\
      \ctxAssignType{\andCtx{\ctx}{\holeCtx}}{\IMV_2}{\TMk{\TCBool}{\NKMV}}
    }{
      \ctxAssignType{\andCtx{\ctx}{\holeCtx}}{\CAnd{\IMV_1}{\IMV_2}}{\TMk{\TCBool}{\NKMV}}
    }{TAAnd} \and

    \judgment{
      \ctxAssignType{\andCtx{\ctx}{\holeCtx}}{\IMV_1}{\TMk{\TCBool}{\NKMV}} \\
      \ctxAssignType{\andCtx{\ctx}{\holeCtx}}{\IMV_2}{\TMk{\TCBool}{\NKMV}}
    }{
      \ctxAssignType{\andCtx{\ctx}{\holeCtx}}{\COr{\IMV_1}{\IMV_2}}{\TMk{\TCBool}{\NKMV}}
    }{TAOr}
  \end{mathpar} \\
  %
  Pairs:
  \begin{mathpar}
    \judgment{
      \ctxAssignType{\andCtx{\ctx}{\holeCtx}}{\IMV_1}{\TMk{\TCMV}{\NKMV}} \\
      \ctxAssignType{\andCtx{\ctx}{\holeCtx}}{\IMV_2}{\TMk{\TCMV'}{\NKMV}}
    }{
      \ctxAssignType{\andCtx{\ctx}{\holeCtx}}{\CPair{\IMV_1}{\IMV_2}}{\TMk{\TCPair{\TCMV}{\TCMV'}}{\NKMV}}
    }{TAPair} \\

    \judgment{
      \ctxAssignType{\andCtx{\ctx}{\holeCtx}}{\IMV}{\TMk{\TCPair{\TCMV}{\TCMV'}}{\NKMV}}
    }{
      \ctxAssignType{\andCtx{\ctx}{\holeCtx}}{\CProjL{\IMV}}{\TMk{\TCMV}{\NKMV}} \\
    }{TAFst} \and

    \judgment{
      \ctxAssignType{\andCtx{\ctx}{\holeCtx}}{\IMV}{\TMk{\TCPair{\TCMV}{\TCMV'}}{\NKMV}}
    }{
      \ctxAssignType{\andCtx{\ctx}{\holeCtx}}{\CProjR{\IMV}}{\TMk{\TCMV'}{\NKMV}} \\
    }{TASnd}
  \end{mathpar} \\
  %
  Application:
  \begin{mathpar}
    \judgment{
      \ctxAssignType{\andCtx{\ctx}{\holeCtx}}{\IMV_1}{\TMk{(\TCFun{\TMk{\TCMV}{\NKMV}}{\TMV})}{\KNC}} \\
      \ctxAssignType{\andCtx{\ctx}{\holeCtx}}{\IMV_2}{\TMk{\TCMV}{\NKMV}}
    }{
      \ctxAssignType{\andCtx{\ctx}{\holeCtx}}{\CAp{\IMV_1}{\IMV_2}}{\TMV}
    }{TAAp1} \and

    \judgment{
      \ctxAssignType{\andCtx{\ctx}{\holeCtx}}{\IMV_1}{\TMk{(\TCFun{\TMk{\TCMV_1}{\NKMV_1}}{\TMk{\TCMV_2}{\NKMV_2}})}{\KNI}} \\
      \ctxAssignType{\andCtx{\ctx}{\holeCtx}}{\IMV_2}{\TMk{\TCMV_1}{\NKMV_1}}
    }{
      \ctxAssignType{\andCtx{\ctx}{\holeCtx}}{\CAp{\IMV_1}{\IMV_2}}{\TMk{\TCMV_2}{\KNC}}
    }{TAAp2}
  \end{mathpar} \\
  %
  \caption{Type assignment for LIR composites}
  \label{fig:lir-ta-comp}
\end{figure}

\begin{figure} \ContinuedFloat
  \judgbox{\ctxAssignType{\andCtx{\ctx}{\holeCtx}}{\CMV}{\TMV}}
    $\CMV$ is assigned type $\TMV$ \\

  Wrapping:
  \begin{mathpar}
    \judgment{
      \ctxAssignType{\andCtx{\ctx}{\holeCtx}}{\IMV}{\TMk{\TCMV}{\KNC}}
    }{
      \ctxAssignType{\andCtx{\ctx}{\holeCtx}}{\CWrapIntoNI{\IMV}}{\TMk{\TCMV}{\KNI}}
    }{TAWrapIntoNI} \and

    \judgment{
      \ctxAssignType{\andCtx{\ctx}{\holeCtx}}{\IMV}{\TMk{\TCMV}{\NKMV}}
    }{
      \ctxAssignType{\andCtx{\ctx}{\holeCtx}}{\CWrapIntoII{\IMV}}{\TMk{\TCMV}{\KII}}
    }{TAWrapIntoII}
  \end{mathpar} \\
  %
  Casting:
  \begin{mathpar}
    \judgment{
      \ctxAssignType{\andCtx{\ctx}{\holeCtx}}{\IMV}{\TMk{\TCMV}{\NKMV}}
    }{
      \ctxAssignType{\andCtx{\ctx}{\holeCtx}}{\CEmbed{\IMV}}{\TMk{\TCHole}{\KNI}}
    }{TAEmbed} \and

    \judgment{
      \ctxAssignType{\andCtx{\ctx}{\holeCtx}}{\IMV}{\TMk{\TCMV}{\KNI}}
    }{
      \ctxAssignType{\andCtx{\ctx}{\holeCtx}}{\CProj{\IMV}{\TCMV'}}{\TMk{\TCMV'}{\KII}}
    }{TAProj}
  \end{mathpar} \\
  %
  Binding:
  \begin{mathpar}
    \judgment{
      \ctxAssignType{\andCtx{\ctx}{\holeCtx}}{\CMV}{\TMV} \\
      \ctxAssignType{\andCtx{\extendCtx{\ctx}{x}{\TMV}}{\holeCtx}}{\CMV'}{\TMV'}
    }{
      \ctxAssignType{\andCtx{\ctx}{\holeCtx}}{\CLet{x}{\CMV} \CIn \CMV'}{\TMV'}
    }{TALet}
  \end{mathpar} \\
  %
  $\KII$ elimination:
  \begin{mathpar}
    \judgment{
      \ctxAssignType{\andCtx{\ctx}{\holeCtx}}{\IMV}{\TMk{\TCMV}{\KII}} \\\\
      \ctxAssignType{\andCtx{\extendCtx{\ctx}{x}{\TMk{\TCMV}{\KNC}}}{\holeCtx}}{\CMV}{\TMV} \\
      \ctxAssignType{\andCtx{\extendCtx{\ctx}{x'}{\TMk{\TCMV}{\KNI}}}{\holeCtx}}{\CMV'}{\TMV}
    }{
      \ctxAssignType{\andCtx{\ctx}{\holeCtx}}{
        \CCaseCompleteWith{\IMV}
          ~\CCaseCompleteBr{x}{\CMV}
          ~\CCaseCompleteBr{x'}{\CMV'}
        }{\TMV}
    }{TACaseComplete}
  \end{mathpar}
  %
  \caption{Type assignment for LIR composites (cont.)}
  \label{fig:lir-ta-comp-cont}
\end{figure}

\subsubsection{Operational semantics}
\label{sec:lir-operational-semantics}
We attempt to give a small-step operational semantics for the lower intermediate representation.

\begin{figure}[htb!]
  \caption{Small-step operational semantics for LIR}
  \label{fig:lir-ssos}
\end{figure}

\subsection{Explicitization}
\label{sec:explicitization}

\begin{figure}[htb!]
  \caption{LIR function-local completeness analysis}
  \label{fig:lir-completeness-analysis-local}
\end{figure}
   
\end{document}
