\documentclass[index.tex]{subfiles}

\begin{document}
\section{Formalization}
\label{formalization}
We try to formulate a calculus for the compilation of functional programs with holes. This
formalization is very much in-progress and likely does not reflect the current status of the
implementation. Ideally, once the calculus is stable, we should like to prove that these
translations preserve the semantics of the source language.

\subsection{Overview}
Compilation begins with the Hazel internal language and passes through a number of stages.

\subsubsection{Transformation}
The implementation inserts a first pass (that is quite unnecessary and whose processes probably
should be folded into a subsequent pass) which cleans up the internal language structure to be
easier to compile.

\subsubsection{Sequentialization}
\emph{Sequentialization} (see \Cref{sec:sequentialization}, currently called ``linearization'' in
the codebase) produces a sequential form resembling monadic normal form \cite{danvy2003}. This form
is called the \emph{middle intermediate representation} (see \Cref{sec:mir}).

\begin{thought}
  Maybe in this phase we should compile evaluation logic concerning casts into explicit operations?
  This would significantly change how casts should be represented.
\end{thought}

\subsubsection{Explicitization}
\emph{Explicitization} (see \Cref{sec:explicitization}) performs completeness analysis (see
\Cref{sec:completeness-analysis}) on the middle intermediate representation and produces a
\emph{lower intermediate representation} (see \Cref{sec:lir}).

\subfile{03-01-middle-ir}
\subfile{03-02-lower-ir}

\end{document}
