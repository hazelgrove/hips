\documentclass[12pt,letterpaper]{article}

\usepackage{mathpartir}
\usepackage{ latexsym }
\usepackage{stmaryrd}

%% Joshua Dunfield macros
\def\OPTIONConf{1}%
\usepackage{joshuadunfield}


\newcommand{\elabAna}[6]{#1 \vdash #2 \Leftarrow #3 \leadsto #4 : #5 \dashv #6}
\newcommand{\elabSyn}[5]{#1 \vdash #2 \Rightarrow #3 \leadsto #4 \dashv #5}
\newcommand{\patAna}[4]{#1 \vdash #2 \leftarrow #3 \dashv #4}
\newcommand{\kconsistent}[2]{#1 \sim #2}
\newcommand{\kconsubkind}[3]{#1 \vdash #2 <\sim #3}
\newcommand{\kindAssign}[3]{#1 \vdash #2 : #3}
\newcommand{\kformation}[2]{#1 \vdash #2 \textt{ kind} }
\newcommand{\kequiv}[3]{#1 \vdash #2 \equiv #3}
\newcommand{\kcequiv}[4]{#1 \vdash #2 \equiv #3 : #4}
\newcommand{\ksubkind}[3]{#1 \vdash #2 <:: #3}
\newcommand{\tyvarValid}[1]{#1 \textt{ valid} }
\newcommand{\tyvarNotValid}[1]{#1 \textt{ }\lnot\textt{valid} }
\newcommand{\dexpTypeAssign}[3]{#1 \vdash #2 : #3}


\newcommand{\hPhi}{\Phi}
\newcommand{\hGamma}{\Gamma}
\newcommand{\EmptyhPhi}{\cdot}
\newcommand{\EmptyDelta}{\cdot}

\newcommand{\hexp}{\hat{e}}
\newcommand{\dexp}{e}
\newcommand{\htau}{\hat{\tau}}
\newcommand{\hkappa}{\kappa}
\newcommand{\dtau}{\tau}
\newcommand{\hrho}{\hat{\rho}}

\newcommand{\Ty}{\textt{Ty}}
\newcommand{\KHole}{\textt{KHole}}
\newcommand{\KSing}[1]{\textt{S(}#1{\textt{)}}}
\newcommand{\hlist}[1]{\textt{list(}#1\textt{)}}

\newcommand{\llparenthesiscolor}{\textcolor{violet}{\llparenthesis}}
\newcommand{\rrparenthesiscolor}{\textcolor{violet}{\rrparenthesis}}
\newcommand{\hthole}[1]{\llparenthesiscolor\rrparenthesiscolor^{#1}}
\newcommand{\hhole}[2]{\llparenthesiscolor#1\rrparenthesiscolor^{#2}}
\newcommand{\kehole}{\llparenthesiscolor\rrparenthesiscolor}

\newcommand{\idof}[1]{\mathsf{id}(#1)}
\newcommand{\Dbinding}[3]{#1 :: #3[#2]}

\newcommand{\htdefine}[3]{\textt{type }#1 \textt{ = } #2\textt{ in }#3}
\newcommand{\dtdefine}[2]{\textt{type bindings } #1 \textt{ in } #2}

\begin{document}

\begin{figure}[t]
	$\arraycolsep=4pt\begin{array}{rllllll}
			\mathsf{BinOp}               & \oplus & ::= &
			\textt{Product} ~\vert~ \textt{Sum} ~\vert~ \textt{Arrow}                                          \\
			\mathsf{Kind}                & \kappa & ::= &
			\Ty ~\vert~ \KHole ~\vert~ \KSing{\tau}                                                            \\
			\mathsf{Constant Types}      & c      & ::= &
			\textt{Int} ~\vert~ \textt{Float} ~\vert~ \textt{Bool}                                             \\
			\mathsf{User HTyp}           & \htau  & ::= &
			c ~\vert~ \htau_1 \oplus \htau_2 ~\vert~ \hlist{\htau} ~\vert~ \hthole{u} ~\vert~ \hhole{\htau}{u} \\
			\mathsf{Internal HTyp}       & \dtau  & ::= &
			c ~\vert~ \dtau_1 \oplus \dtau_2 ~\vert~ \hlist{\dtau} ~\vert~ \hthole{u} ~\vert~ \hhole{\dtau}{u} \\
			\mathsf{Type Vars}           & t      &     &                                                      \\
			\mathsf{User Type Pattern}   & \hrho  & ::= &
			t ~\vert~ \hthole{u} ~\vert~ \hhole{t}{u}                                                          \\
			\mathsf{User Expression}     & \hexp  & ::= &
			\htdefine{\hrho}{\htau}{\hexp} ~\vert~ elided                                                      \\
			\mathsf{Internal Expression} & \dtau  & ::= &
			\dtdefine{\hPhi}{\dexp} ~\vert~ elided                                                             \\
		\end{array}$
\end{figure}


\begin{minipage}{\linewidth}
	\judgbox
	{\kconsubkind{\Delta;\hPhi}{\hkappa_1}{\hkappa_2}}
	{$\hkappa_1$ is a consistent subkind of $\hkappa_2$}
	\begin{mathpar}
		\inferrule[KCHoleL]{ }{
			\kconsubkind{\Delta;\hPhi}{\KHole}{\hkappa}
		}
		\and
		\inferrule[KCHoleR]{ }{
			\kconsubkind{\Delta;\hPhi}{\hkappa}{\KHole}
		}
		\and
		\inferrule[KCRespectEquiv]{
			\kequiv{\Delta;\hPhi}{\hkappa_1}{\hkappa_2} \\
		} {
			\kconsubkind{\Delta;\hPhi}{\hkappa_1}{\hkappa_2}
		}
		\and
		\inferrule[KCSubsumption]{
			\kindAssign{\Delta;\hPhi}{\dtau}{\Ty}
		} {
			\kconsubkind{\Delta;\hPhi}{\KSing{\dtau}}{\Ty}
		}
	\end{mathpar}
\end{minipage}
\\
\\


\begin{minipage}{\linewidth}
	\judgbox
	{\tyvarValid{t}}
	{$t$ is a valid type variable} \; \\
	$t$ is valid if it is not a builtin-type or keyword, begins with an alpha char or underscore, and only contains alphanumeric characters, underscores, and primes.
\end{minipage}
\\
\\


\begin{minipage}{\linewidth}
	\judgbox
	{\kformation{\Delta;\hPhi}{\hkappa}}
	{$\hkappa$ forms a kind} \;\\
	\begin{mathpar}
		\inferrule[KFSing]{
			\kindAssign{\Delta;\hPhi}{\dtau}{\Ty}
		} {
			\kformation{\Delta;\hPhi}{\KSing{\dtau}}
		}
	\end{mathpar}
\end{minipage}
\\
\\

\begin{minipage}{\linewidth}
	\judgbox
	{\kequiv{\Delta;\hPhi}{\hkappa_1}{\hkappa_2}}
	{$\hkappa_1$ is equivalent to $\hkappa_2$} \;\\
	\begin{mathpar}
		\inferrule[KERefl]{ } {
			\kequiv{\Delta;\hPhi}{\hkappa}{\hkappa}
		}
		\and
		\inferrule[KESymm]{
			\kequiv{\Delta;\hPhi}{\hkappa_1}{\hkappa_2}
		} {
			\kequiv{\Delta;\hPhi}{\hkappa_2}{\hkappa_1}
		}
		\and
		\inferrule[KETrans]{
			\kequiv{\Delta;\hPhi}{\hkappa_1}{\hkappa_2} \\
			\kequiv{\Delta;\hPhi}{\hkappa_2}{\hkappa_3}
		} {
			\kequiv{\Delta;\hPhi}{\hkappa_1}{\hkappa_3}
		}
		\and
		\inferrule[KESingEquiv]{
			\kcequiv{\Delta;\hPhi}{\dtau_1}{\dtau_2}{\Ty}
		} {
			\kequiv{\Delta;\hPhi}{\KSing{\dtau_1}}{\KSing{\dtau_2}}
		}
	\end{mathpar}
\end{minipage}
\\
\\

\begin{minipage}{\linewidth}
	\judgbox
	{\kindAssign{\Delta;\hPhi}{\dtau}{\hkappa}}
	{$\dtau$ is assigned kind $\hkappa$}
	\begin{mathpar}
		\inferrule[KAConst]{ }{
			\kindAssign{\Delta;\hPhi}{c}{\Ty}
		}
		\and
		\inferrule[KAVar]{
			t : \hkappa \in \hPhi
		} {
			\kindAssign{\Delta;\hPhi}{t}{\hkappa}
		}
		\and
		\inferrule[KABinOp]{
			\kindAssign{\Delta;\hPhi}{\dtau_1}{\Ty}
			\kindAssign{\Delta;\hPhi}{\dtau_2}{\Ty}
		} {
			\kindAssign{\Delta;\hPhi}{\dtau_1 \oplus \dtau_2}{\Ty}
		}
		\and
		\inferrule[KAList]{
			\kindAssign{\Delta;\hPhi}{\dtau}{\Ty}
		} {
			\kindAssign{\Delta;\hPhi}{\hlist{\dtau}}{\Ty}
		}
		\and
		\inferrule[KAEHole]{
			\Dbinding{u}{\hPhi'}{\hkappa} \in \Delta \\
			\kindAssign{\Delta;\hPhi}{\sigma}{\hPhi'}
		} {
			\kindAssign{\Delta;\hPhi}{\hthole{u}_{\sigma}}{\hkappa}
		}
		\and
		\inferrule[KANEHole]{
			\kindAssign{\Delta;\hPhi}{\dtau}{\hkappa'} \\
			\Dbinding{u}{\hPhi'}{\hkappa} \in \Delta \\
			\kindAssign{\Delta;\hPhi}{\sigma}{\hPhi'}
		} {
			\kindAssign{\Delta;\hPhi}{\hhole{\dtau}{u}_{\sigma}}{\hkappa}
		}
		\and
		\inferrule[KASelfRecognition]{
			\kindAssign{\Delta;\hPhi}{\dtau}{\Ty}
		}{
			\kindAssign{\Delta;\hPhi}{\dtau}{\KSing{\dtau}}
		}
		\and
		\inferrule[KASubkind]{
			\kindAssign{\Delta;\hPhi}{\dtau}{\hkappa_1} \\
			\kconsubkind{\Delta;\hPhi}{\hkappa_1}{\hkappa_2}
		} {
			\kindAssign{\Delta;\hPhi}{\dtau}{\hkappa_2}
		}

	\end{mathpar}
\end{minipage}
\\
\\




\begin{minipage}{\linewidth}
	\judgbox
	{\kcequiv{\Delta;\hPhi}{\dtau_1}{\dtau_2}{\hkappa}}
	{$\dtau_1$ is equivalent to $\dtau_2$ and has kind $\hkappa_2$} \;\\
	\begin{mathpar}
		\inferrule[KCERefl]{ } {
			\kcequiv{\Delta;\hPhi}{\dtau}{\dtau}{\hkappa}
		}
		\and
		\inferrule[KCESymm]{
			\kcequiv{\Delta;\hPhi}{\dtau_1}{\dtau_2}{\hkappa}
		} {
			\kcequiv{\Delta;\hPhi}{\dtau_2}{\dtau_1}{\hkappa}
		}
		\and
		\inferrule[KCETrans]{
			\kcequiv{\Delta;\hPhi}{\dtau_1}{\dtau_2}{\hkappa} \\
			\kcequiv{\Delta;\hPhi}{\dtau_2}{\dtau_3}{\hkappa}
		} {
			\kcequiv{\Delta;\hPhi}{\dtau_1}{\dtau_3}{\hkappa}
		}
		\and
		\inferrule[KCESingEquiv]{
			\kindAssign{\Delta;\hPhi}{\dtau_1}{\KSing{\dtau_2}}
		} {
			\kcequiv{\Delta;\hPhi}{\dtau_1}{\dtau_2}{\Ty}
		}
	\end{mathpar}
\end{minipage}
\\
\\

\begin{minipage}{\linewidth}
	\judgbox
	{\elabSyn{\hPhi}{\htau}{\hkappa}{\dtau}{\Delta}}
	{$\htau$ synthesizes kind $\hkappa$ and elaborates to $\dtau$}
	\begin{mathpar}
		\inferrule[TElabSConst]{ }{
			\elabSyn{\hPhi}{c}{\Ty}{c}{\EmptyDelta}
		}
		\and
		\inferrule[TElabSBinOp]{
			\elabAna{\hPhi}{\htau_1}{\Ty}{\dtau_1}{\Ty}{\Delta_1} \\
			\elabAna{\hPhi}{\htau_2}{\Ty}{\dtau_2}{\Ty}{\Delta_2}
		}{
			\elabSyn{\hPhi}{\htau_1 \oplus \htau_2}{\Ty}{\dtau_1 \oplus \dtau_2}{\Delta_1 \cup \Delta_2}
		}
		\and
		\inferrule[TElabSList]{
			\elabAna{\hPhi}{\htau}{\Ty}{\dtau}{\Ty}{\Delta}
		} {
			\elabSyn{\hPhi}{\hlist{\htau}}{\Ty}{\hlist{\dtau}}{\Delta}
		}
		\and
		\inferrule[TElabSVar]{
			t : \hkappa \in \hPhi
		} {
			\elabSyn{\hPhi}{t}{\hkappa}{t}{\EmptyDelta}
		}
		\and
		\inferrule[TElabSUVar]{
		t \not\in \hPhi
		} {
		\elabSyn{\hPhi}{t}{\KHole}{\hhole{t}{u}_{\idof{\hPhi}}}{\Dbinding{u}{\hPhi}{\kehole} }
		}
		\and
		\inferrule[TElabSHole]{ } {
		\elabSyn{\hPhi}{\hthole{u}}{\KHole}{\hthole{u}_{\idof{\hPhi}}}{\Dbinding{u}{\hPhi}{\kehole}}
		}
		\and
		\inferrule[TElabSNEHole]{
		\elabSyn{\hPhi}{\htau}{\hkappa}{\dtau}{\Delta}
		} {
		\elabSyn{\hPhi}{\hhole{\htau}{u}}{\KHole}{\hhole{\dtau}{u}_{\idof{\hPhi}}}{\Delta,\Dbinding{u}{\hPhi}{\kehole}}
		}
	\end{mathpar}
\end{minipage}
\\
\\
\begin{minipage}{\linewidth}
	\judgbox
	{\elabAna{\hPhi}{\htau}{\hkappa_1}{\dtau}{\hkappa_2}{\Delta}}
	{$\htau$ analyzes against kind $\hkappa_1$ and
		elaborates to $\dtau$ of consistent kind $\hkappa_2$}
	\begin{mathpar}
		\inferrule[TElabASubsume]{
			\htau \neq t \text{ where } t \not\in \hPhi \\
			\htau \neq \hthole{u} \\
			\htau \neq \hhole{\htau'}{u} \\
			\elabSyn{\hPhi}{\htau}{\hkappa'}{\dtau}{\Delta} \\
			\kconsistent{\hkappa}{\hkappa'}
		}{
			\elabAna{\hPhi}{\htau}{\hkappa}{\dtau}{\hkappa'}{\Delta}
		}
		\and
		\inferrule[TElabAUVar]{
		t \not\in \hPhi
		} {
		\elabAna{\hPhi}{t}{\KHole}{\hhole{t}{u}_{\idof{\hPhi}}}{\KHole}{\Dbinding{u}{\hPhi}{\kehole} }
		}
		\and
		\inferrule[TElabAEHole]{ } {
		\elabAna{\hPhi}{\hthole{u}}{\hkappa}{\hthole{u}_{\idof{\hPhi}}}{\hkappa}{\Dbinding{u}{\hPhi}{\kappa}}
		}
		\and
		\inferrule[TElabANEHole]{
		\elabSyn{\hPhi}{\htau}{\hkappa'}{\dtau}{\Delta}
		} {
		\elabAna{\hPhi}{\hhole{\htau}{u}}{\hkappa}{\hhole{\dtau}{u}_{\idof{\hPhi}}}{\hkappa}{\Delta,\Dbinding{u}{\hPhi}{\kappa}}
		}

	\end{mathpar}
\end{minipage}
\\
\\

\begin{minipage}{\linewidth}
	\judgbox
	{\patAna{\Delta_1;\hPhi_1}{\hrho}{\dtau}{\hPhi_2;\Delta_2}}
	{$\hrho$ analyzes against $\dtau$ yielding new tyvar and hole bindings}
	\begin{mathpar}
		\inferrule[RESVar]{
			\tyvarValid{t} \\
			\kindAssign{\Delta;\hPhi}{\dtau}{\hkappa}
		}{
			\patAna{\Delta;\hPhi}{t}{\dtau}{t :: \hkappa;\EmptyDelta}
		}
		\and
		\inferrule[RESEHole]{ }{
			\patAna{\Delta;\hPhi}{\hthole{u}}{\dtau}{\EmptyhPhi;\Dbinding{u}{\hPhi}{\kehole}}
		}
		\and
		\inferrule[RESVarHole]{
			\tyvarNotValid{t}
		}{
			\patAna{\Delta;\hPhi}{\hhole{t}{u}}{\dtau}{\EmptyhPhi;\Dbinding{u}{\hPhi}{\kehole}}
		}
	\end{mathpar}
\end{minipage}
\\
\\

\begin{minipage}{\linewidth}
	\judgbox
	{\elabSyn{\hGamma;\hPhi}{\hexp}{\htau}{\dexp}{\Delta}}
	{$\hexp$ synthesizes type $\dtau$ and elaborates to $\dexp$}
	\begin{mathpar}
		\inferrule[ESDefine]{
			\elabSyn{\hPhi_1}{\htau}{\hkappa}{\dtau}{\Delta_1} \\
			\patAna{\Delta_1;\hPhi_1}{\hrho}{\dtau}{\hPhi_2;\Delta_2} \\
			\elabSyn{\hGamma;\hPhi_1 \cup \hPhi_2}{\hexp}{\dtau_1}{\dexp}{\Delta_3}
		}{
			\elabSyn{\hGamma;\hPhi_1}{\htdefine{\hrho}{\htau}{\hexp}}{\dtau_1}{\dtdefine{\hPhi_2}{\dexp}}{\Delta_1 \cup \Delta_2 \cup \Delta_3}
		}


	\end{mathpar}
\end{minipage}

\begin{minipage}{\linewidth}
	\judgbox
	{\dexpTypeAssign{\hGamma;\hPhi}{\dexp}{\dtau}}
	{$\dexp$ is assigned type $\dtau$}
	\begin{mathpar}
		\inferrule[DEDefine]{
			\dexpTypeAssign{\hGamma;\hPhi_1 \cup \hPhi_2}{\dexp}{\dtau}
		}{
			\dexpTypeAssign{\hGamma;\hPhi_1}{\dtdefine{\hPhi_2}{\dexp}}{\dtau}
		}


	\end{mathpar}
\end{minipage}









\end{document}
