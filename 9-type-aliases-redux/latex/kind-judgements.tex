%%%%%%%%%%%%%%%%%%%%%%%%%%%%%%%%%%%%%%%%%%%%%%%%%%%%%%%%%%%%
% standard header
    \documentclass[12pt]{article}
    \usepackage[margin=2cm]{geometry}
    \usepackage[utf8x]{inputenc} % for utf8
    \usepackage{titlesec} % see below
    \usepackage{xcolor} % \color \textcolor
    \usepackage{hyperref} % \href
    \usepackage[normalem]{ulem} % normalem retains \emph as italic
    % \uline \uuline \uwave \sout
    \usepackage{enumitem}
    % \begin{enumerate}[label=\Alpha*.]
    % \begin{enumerate}[label=\roman*.]
    % \begin{itemize}
    \usepackage{graphicx}
    % \includegraphicswidth=0.5\textwidth,trim=[0cm 0cm 0cm 0cm,clip]{file.png}
    \usepackage{pdfpages} % \includepdf[pages=1]{file.pdf}
    % for prose
    %\usepackage[doublespacing]{setspace}
    \usepackage{csquotes} % \blockquote
    \usepackage{xparse} % \NewDocumentCommand{\name}{O{#1 default}mO{#3 default}}{def}
    %%%%%%%%%%%%%%%%%%%%%%%%%%%%%%%%%%%%%%%%%%%%%%%%%%%%%%%%%%%%
    \renewcommand{\rmdefault}{cmr}
    \renewcommand{\sfdefault}{cmss}
    \renewcommand{\ttdefault}{cmtt}
    \renewcommand{\familydefault}{\rmdefault}
    \setlength{\titlewidth}{\textwidth}

    \titlespacing{\section}{0pt}{0pt}{0pt}
    \titlespacing{\subsection}{0pt}{0pt}{0pt}
    \titlespacing{\subsubsection}{0pt}{0pt}{0pt}
    \titlespacing{\paragraph}{0pt}{0pt}{0pt}
    % https://www.overleaf.com/learn/latex/How_to_write_a_LaTeX_class_file_and_design_your_own_CV_(Part_1)
    % \titleformat{command}[shape]{format}{label}{sep}{before-code}[after-code]
    \titleformat{\section}         % Customise the \section command
        [hang]
        {\Large\scshape\raggedright} % Make the \section headers large (\Large),
                                   % small capitals (\scshape) and left aligned (\raggedright)
        {}{0em}                      % Can be used to give a prefix to all sections, like 'Section ...'
        {}                           % Can be used to insert code before the heading
        [\titlerule]                 % Inserts a horizontal line after the heading
    \titleformat{\subsection}
        [hang]
        {\large\scshape\raggedright}
        {}{0em}
        {}
        []
%%%%%%%%%%%%%%%%%%%%%%%%%%%%%%%%%%%%%%%%%%%%%%%%%%%%%%%%%%%%
% packages
    \usepackage{hejohns-hazel}
% commands
    \newcommand*{\vt}{~\vert~}
% get rid of 10-modules stuff
    \renewcommand*{\TypVarCtx}[1][]{\Phi}
    \renewcommand*{\ModVarCtx}[1][]{}
    \renewcommand*{\SigVarCtx}[1][]{}
%%%%%%%%%%%%%%%%%%%%%%%%%%%%%%%%%%%%%%%%%%%%%%%%%%%%%%%%%%%%
\pagenumbering{gobble}
\nonfrenchspacing
%\frenchspacing % when monospaced
\begin{document}
\title{Hazel Phi: 9-type-aliases}
\author{}
\date{\today}
\maketitle
\section{syntax}
    \begin{longtable}{RCRL}
        \textrm{BinOp} & \binop[][] & ::=
                  & \ProdType \vt \SumType \vt \FunType \\
        \textrm{Kind} & \knd & ::=
                      & \Type \vt \KHole \vt \SKind \vt \DepFunKind \\
        \textrm{Base Types} & \bse & ::=
                            & \Int \vt \Float \vt \Bool \\
        \textrm{User Types} & \utyp & ::=
                            & \utypvar \vt \bse \vt \ubinop \vt \ETypeHole \vt \NETypeHole[\utyp] \vt \TypeFun[][\Type][\utyp] \vt \TypeAp[\utyp[1]][\utyp[2]] \\
        \textrm{Internal Types} & \typ & ::=
                            & \typvar \vt \bse \vt \binop \vt \ETypeHole \vt \NETypeHole \vt \TypeFun \vt \TypeAp \\
        \textrm{Type Pattern} & \\
        \textrm{User Expression} & \\
        \textrm{Internal Expression} & \\
    \end{longtable}
\section{Declaratives}
\subsection*{}
    \begin{minipage}{\textwidth}
        \judgbox{\ConsistentSubKind{\knd[1]}{\knd[2]}}{$\knd[1]$ is a consistent subkind of $\knd[2]$}
        \begin{mathpar}
            \inferrule{\KindWellFormed}{\ConsistentSubKind{\KHole}{\knd}}
            \and
            \inferrule{\KindWellFormed}{\ConsistentSubKind{\knd}{\KHole}}
            \\
            \inferrule{\KindEquiv{\knd[1]}{\knd[2]}}{\ConsistentSubKind{\knd[1]}{\knd[2]}}
            \and
            \inferrule{\KindEquiv{\knd[1]}{\knd[3]} \\ \ConsistentSubKind{\knd[3]}{\knd[2]}}{\ConsistentSubKind{\knd[1]}{\knd[2]}}
            \and
            \inferrule{\WellFormedAtKind{\typ}}{\ConsistentSubKind{\SKind}{\knd}}
            \and
            \inferrule{\ConsistentSubKind{\knd[1]}{\knd[2]} \\ \TypeEquivAtKind{\typ[1]}{\typ[2]}[\knd[1]]}{\ConsistentSubKind{\SKind[\knd[1]][\typ[1]]}{\SKind[\knd[2]][\typ[2]]}}
            \and
            \inferrule{\ConsistentSubKind{\knd[3]}{\knd[1]} \\ \ConsistentSubKind[\HoleCtx\TypVarCtx,\typvar\KndAssump\knd[3]]{\knd[2]}{\knd[4]}}{\ConsistentSubKind{\DepFunKind}{\DepFunKind[][\knd[3]][\knd[4]]}}
        \end{mathpar}
    \end{minipage}
\subsection*{}
    \begin{minipage}{\textwidth}
        \judgbox{\KindEquiv{\knd[1]}{\knd[2]}}{$\knd[1]$ is equivalent to $\knd[2]$}
        \begin{mathpar}
            \inferrule{\KindWellFormed}{\KindEquiv{\knd}{\knd}}
            \and
            \inferrule{\KindEquiv{\knd[2]}{\knd[1]}}{\KindEquiv{\knd[1]}{\knd[2]}}
            \and
            \inferrule{\KindEquiv{\knd[1]}{\knd[3]} \\ \KindEquiv{\knd[3]}{\knd[2]}}{\KindEquiv{\knd[1]}{\knd[2]}}
            \and
            \inferrule{\TypeEquivAtKind{\typ[1]}{\typ[2]}}{\KindEquiv{\SKind[\knd][\typ[1]]}{\SKind[\knd][\typ[2]]}}
            \and
            \inferrule{\WellFormedAtKind{\typ}[\SKind[\knd][\typ[1]]]}{\KindEquiv{\SKind[\SKind[\knd][\typ[1]]][\typ]}{\SKind[\knd][\typ[1]]}}
            \and
            \inferrule{\WellFormedAtKind{\typ}[\DepFunKind]}{\KindEquiv{\SKind[\DepFunKind][\typ]}{\DepFunKind[][\knd[1]][\SKind[\knd[2]][\TypeAp[\typ][\typvar]]]}}
            \and
            \inferrule{\KindEquiv{\knd[1]}{\knd[2]} \\ \KindEquiv[\HoleCtx\TypVarCtx[],\typvar\KndAssump\knd[1]]{\knd[3]}{\knd[4]}}{\KindEquiv{\DepFunKind[][\knd[1]][\knd[2]]}{\DepFunKind[][\knd[3]][\knd[4]]}}
        \end{mathpar}
    \end{minipage}
\subsection*{}
    \begin{minipage}{\textwidth}
        \judgbox{\TypeEquivAtKind{\typ[1]}{\typ[2]}}{$\typ[1]$ is equivalent to $\typ[2]$ at kind $\knd$}
        \begin{mathpar}
            \inferrule{\WellFormedAtKind{\typ[1]}[\SKind[\knd][\typ[2]]]}{\TypeEquivAtKind{\typ[1]}{\typ[2]}[\knd]}
            \\
            \mprset{myfraction=\inferruledotfrac}
            \inferrule{\WellFormedAtKind{\typ}}{\TypeEquivAtKind{\typ}{\typ}}
            \and
            \inferrule{\TypeEquivAtKind{\typ[2]}{\typ[1]}}{\TypeEquivAtKind{\typ[1]}{\typ[2]}}
            \and
            \inferrule{\TypeEquivAtKind{\typ[1]}{\typ[3]} \\ \TypeEquivAtKind{\typ[3]}{\typ[1]}}{\TypeEquivAtKind{\typ[1]}{\typ[2]}}
            \and
            \inferrule{\TypeEquivAtKind{\typ[1]}{\typ[3]}[\Type] \\ \TypeEquivAtKind{\typ[2]}{\typ[4]}[\Type]}{\TypeEquivAtKind{\binop}{\binop[\typ[3]][\typ[4]]}[\Type]}
            \and
            \inferrule{\KindEquiv{\knd[1]}{\knd[2]} \\ \TypeEquivAtKind[\HoleCtx\TypVarCtx[],\typvar\KndAssump\knd[1]]{\typ[1]}{\typ[2]}}{\TypeEquivAtKind{\TypeFun[][\knd[1]][\typ[1]]}{\TypeFun[][\knd[2]][\typ[2]]}[\DepFunKind[][\knd[1]][\knd]]}
            \and
            \inferrule{\TypeEquivAtKind{\typ[1]}{\typ[3]}[\DepFunKind] \\ \TypeEquivAtKind{\typ[2]}{\typ[4]}[\knd[1]]}{\TypeEquivAtKind{\TypeAp}{\TypeAp[\typ[3]][\typ[4]]}[\subst{\typ[1]}{\typvar}{\knd[2]}]}
            \and
            \inferrule{\WellFormedAtKind{\typ[1]}[\DepFunKind[][\knd[1]][\knd[3]]] \\ \WellFormedAtKind{\typ[2]}[\DepFunKind[][\knd[1]][\knd[4]]] \\ \TypeEquivAtKind[\HoleCtx\TypVarCtx,\typvar\KndAssump\knd[1]]{\TypeAp[\typ[1]][\typvar]}{\TypeAp[\typ[2]][\typvar]}[\knd[2]]}{\TypeEquivAtKind{\typ[1]}{\typ[2]}[\DepFunKind]}
            \and
            \inferrule{\TypeEquivAtKind{\typ[1]}{\typ[2]}[\knd[1]] \\ \KindEquiv{\knd[1]}{\knd}}{\TypeEquivAtKind{\typ[1]}{\typ[2]}}
        \end{mathpar}
    \end{minipage}
\subsection*{}
    \begin{minipage}{\textwidth}
        \judgbox{\PrincipalKind{\typ}}{$\typ$ has principal kind $\knd$}
        \begin{mathpar}
            \inferrule{ }{\PrincipalKind{\bse}[\SKind[\Type][\bse]]}
            \and
            \inferrule{\typvar\KndAssump\knd\in\TypVarCtx[]}{\PrincipalKind{\typvar}[\SKind[\knd][\typvar]]}
            \and
            \inferrule{\WellFormedAtKind{\typ[1]}[\Type] \\ \WellFormedAtKind{\typ[2]}[\Type]}{\PrincipalKind{\binop}[\SKind[\Type][\binop]]}
            \and
            \inferrule{\hole\KndAssump\knd\in\HoleCtx[]}{\PrincipalKind{\ETypeHole}}
            \and
            \inferrule{\hole\KndAssump\knd\in\HoleCtx[] \\ \WellFormedAtKind{\typ}[\knd[1]]}{\PrincipalKind{\NETypeHole}}
            \and
            \inferrule{\hole\KndAssump\knd\in\HoleCtx[] \\ \typvar\notin\dom{\TypVarCtx[]}}{\PrincipalKind{\UnboundTypeVar}}
            \and
            \inferrule{\PrincipalKind[\HoleCtx\TypVarCtx[],\typvar\KndAssump\knd[1]]{\typ}[\knd[2]]}{\PrincipalKind{\TypeFun[][\knd[1]]}[\SKind[\DepFunKind][\TypeFun[][\knd[1]]]]}
            \and
            \inferrule{\PrincipalKind{\typ[1]} \\ \MatchedPiKind{\knd} \\ \WellFormedAtKind{\typ[2]}[\knd[1]]}{\PrincipalKind{\TypeAp}[\subst{\typ[2]}{\typvar}{\knd[2]}]}
        \end{mathpar}
    \end{minipage}
\subsection*{}
    \begin{minipage}{\textwidth}
        \judgbox{\MatchedPiKind{\knd}}{$\knd$ has matched $\Pi$-kind $\DepFunKind$}
        \begin{mathpar}
            \inferrule{ }{\MatchedPiKind{\KHole}[\DepFunKind[][\KHole][\KHole]]}
            \and
            \inferrule{\KindEquiv{\knd}{\DepFunKind}}{\MatchedPiKind{\knd}}
        \end{mathpar}
    \end{minipage}
\subsection*{}
    \begin{minipage}{\textwidth}
        \judgbox{\WellFormedAtKind{\typ}}{$\typ$ is well formed at kind $\knd$}
        \begin{mathpar}
            \inferrule{\PrincipalKind{\typ}}{\WellFormedAtKind{\typ}}
            \and
            \inferrule{\WellFormedAtKind{\typ}[\SKind[\knd][\typ[1]]]}{\WellFormedAtKind{\typ}}
            \and
            \inferrule{\PrincipalKind{\typ}[\knd[1]] \\ \ConsistentSubKind{\knd[1]}{\knd}}{\WellFormedAtKind{\typ}}
            \\
            \inferrule{\WellFormedAtKind{\typ[2]}[\SKind[\knd][\typ[1]]]}{\WellFormedAtKind{\typ[1]}[\SKind[\knd][\typ[2]]]}
            \and
            \inferrule{\WellFormedAtKind{\typ[1]}[\SKind[\knd][\typ[3]]] \\ \WellFormedAtKind{\typ[3]}[\SKind[\knd][\typ[2]]]}{\WellFormedAtKind{\typ[1]}[\SKind[\knd][\typ[2]]]}
        \end{mathpar}
    \end{minipage}
\end{document}
